%%%%%%%%%%%%%%%%%%%%%%%%%%%%%%%%%%%%%%%%%%%%%%%%%%%%%%%%%%%%%%%%%%%%%%%%%%%%%%%

\documentclass[12pt,twocolumn,tighten]{aastex63}
%\documentclass[12pt,twocolumn,tighten,trackchanges]{aastex63}
\usepackage{amsmath,amstext,amssymb}
\usepackage[T1]{fontenc}
\usepackage{apjfonts}
\usepackage[figure,figure*]{hypcap}
\usepackage{graphics,graphicx}
\usepackage{hyperref}
\usepackage{natbib}
\usepackage[caption=false]{subfig} % for subfloat
\usepackage{enumitem} % for specific spacing of enumerate
\usepackage{epigraph}

\renewcommand*{\sectionautorefname}{Section} %for \autoref
\renewcommand*{\subsectionautorefname}{Section} %for \autoref

\newcommand{\cn}{NGC\,2516} % cluster name

\newcommand{\numcdipslcs}{671{,}901\ } % drivers.yield_calculation.count_cdips_lcs.py; 2021.01.13
\newcommand{\numsouthernuniqlcs}{483{,}407\ } % drivers.yield_calculation.count_cdips_lcs.py; 2021.01.13

\newcommand{\kms}{\,km\,s$^{-1}$}

%% Reintroduced the \received and \accepted commands from AASTeX v5.2.
%% Add "Submitted to " argument.
\received{---}
\revised{---}
\accepted{---}
\submitjournal{AAS journals.}
\shorttitle{Core \& Halo of NGC\,2516}

\begin{document}

\defcitealias{bouma_wasp4b_2019}{B19}

\title{
  Cluster Difference Imaging Photometric Survey. III.
  The Coeval Halo and Core of NGC\,2516
}

%\suppressAffiliations
%\NewPageAfterKeywords
\correspondingauthor{L.\,G.\,Bouma}
\email{luke@astro.princeton.edu}

\author[0000-0002-0514-5538]{L. G. Bouma}
\affiliation{Department of Astrophysical Sciences, Princeton University, 4 Ivy Lane, Princeton, NJ 08540, USA}

\author[0000-0001-8732-6166]{J. D. Hartman}
\affiliation{Department of Astrophysical Sciences, Princeton University, 4 Ivy Lane, Princeton, NJ 08540, USA}

\author[0000-0002-4265-047X]{J. N. Winn}
\affiliation{Department of Astrophysical Sciences, Princeton University, 4 Ivy Lane, Princeton, NJ 08540, USA}

\author[0000-0001-7204-6727]{G. \'A. Bakos}
\affiliation{Department of Astrophysical Sciences, Princeton University, 4 Ivy Lane, Princeton, NJ 08540, USA}



\begin{abstract}
  foo,bar.
\end{abstract}

\keywords{
	stellar ages (1581),
	young star clusters (1833),
  open clusters (1160),
  stellar rotation (1629)
}

%%%%%%%%%%%%%%%%%%%%%%%%%%%%%%%%%%%%%%%%%%%%%%%%%%%%%%%%%%%%%%%%%%%%%%%%%%%%%%%


\section{Introduction}

In the traditional picture of star formation molecular clouds go
gravitationally unstable, and collapse into little knots.  The knots
produce many stars close to each other.  The resulting ``open star
clusters'' are, as with any first-order guess, spherical.

How does the cluster evolve?  The answer depends
on the total stellar mass.
% What exactly does this mean?
The smallest 90\% of star clusters disperse during the ``embedded
phase'' (CITE).  This is driven mostly by {\it process A} (CITE).  The
more massive clusters make it to maybe 100 Myr. Maybe a bit longer.
Their evaporation is thought to be driven by collisions with molecular
clouds (CITE, Spitzer 1958), and the galactic tide (CITE), and {\it
process A}, and {\it process B}.  Of course, the expected evaporation
time should depend on factors including the mass of the cluster
itself, how many high-mass (O and B) stars form, since their winds
[and maybe supernovae] clear out most of the cloud, and the initial
density of the gas cloud to begin with.

Assuming that the cluster achieves virial equilibrium, stellar fly-bys
then conspire to segregate the stellar mass distribution within the
cluster, evaporating the lowest mass stars soonest (2T+U=0, U=-GM/r,
assuming equipartition of the specific ``thermal energy'' per star,
then kT = 0.5$mv^2$ at lower masses requires higher velocities).  

Identification of the stars that disperse into the galactic field is
an important task for understanding the conditions under which stars
and star clusters form, and for understanding how they subsequently
evolve.  For instance, how does the process of radial migration across
the galactic disk affect cluster dispersal?  Did the Sun form in an
open cluster? If so, how massive was its cluster, and is there any
hope at identifying the stars that formed near the Sun?
Qualitatively, the dispersal of open clusters also provides perhaps
the best ``test case'' for the concept of ``chemical tagging'', also
referred to as ``galactic archaeology'' (CITE).

Outside of the issue of star formation, identifying the remnant halos
of open clusters is important for a separate project: that of
discovering young transiting planets.  Young transiting planets are
hard to find because young stars are rare (CITE), and often reside in
crowded regions of the sky (CITE).  If the halos of nearby star
clusters can be reliably identified, this could expand the census of
nearby sub-Gyr stars by a factor of 2 or even 3 (CITE Meingast).  A
fortuitous benefit of searching for planets in cluster environments is
that issues with stellar crowding are also alleviated.

Recent clustering studies using Gaia have begun to report the
identification of structures that could correspond to low-density
halos of stars that may have evaporated from open clusters (e.g.,
Kounkel+19, Kounkel+20, Meingast+21).  However, different clustering
methods on the Gaia data tend to give different results (CITE: Hunt \&
Reffert 2020).  Employing say Gaussian Mixture Modelling, or any
analogous method that requires ``clusters'' to be ellipses in
spatial/velocity phase-space unsurpsingly yields open clusters that
are roughly elliptical (CITE: Cantat-Gaudin+18). Unsupervised
clustering methods such as HDBScan (e.g., Kounkel+19) have been found
to yield additional structures, particularly in lower density regions
such as the Psc-Eri stream (CITE: Meingast+18, Curtis+19, Newton+21),
but also more generally around many open clusters (Kounkel+19).
Unsupervised approaches that incorporate physical constraints (e.g.,
imposing a maximum velocity dispersion on putative members) yield
similar results, potentially with higher purity (Meingast+21).

We've recently been making TESS light curves of age-dated stars across
the sky, as part of a Cluster Difference Imaging Photometry Survey
(CDIPS, CITE Bouma+19).  Our analysis of Cycle 1 (Sectors 1-13)
yielded light curves of \numsouthernuniqlcs candidate cluster members
in the Southern Ecliptic hemisphere, available on MAST
[\href{https://archive.stsci.edu/hlsp/cdips}{1},\href{https://ui.adsabs.harvard.edu/abs/2019ApJS..245...13B/abstract}{2}].
% TODO: make this number actually quantitative...
Based on rotation periods, $\approx$25\% appear to be bonafide cluster
members. 

As part of a broader project of identifying a large and clean sample
of young stars for a transit search, we take the opportunity of this
paper to answer a rather modest question: in just a single rich
southern open cluster, is the cluster halo coeval with the core?  The
cluster we chose for this analysis was NGC\,2516, since it was young
($\sim$110\,Myr) and close ($d=400$\,pc) enough to facilitate rotation
measurements using TESS, and some degree of spectroscopic analysis.
We want to know: is the halo real? To what extent can we use Gaia
alone to reliably identify age-dated needles in the haystack of boring
field stars?  And more generally, what are the implications for the
evolution of open clusters if they do have halos?

Section~\ref{sec:gaia} presents the astrometric and photometric data
from Gaia, and clarifies our usage of the terms ``core'' and ``halo''.
Section~\ref{sec:agedate} age-dates the halo of NGC\,2516, using TESS
gyrochronology (Section~\ref{sec:rotation}), and lithium depletion
(Section~\ref{sec:lithium}).  In Section~\ref{sec:discussion} we
discuss the implications of this analysis for NGC\,2516 specifically
and stellar spin-down and open cluster evolution generally.
Section~\ref{sec:conclusion} presents our conclusions.


\section{A 250\,pc Halo around a Core?}
\label{sec:gaia}
%
% What is the core? What is the halo?
%

\subsection{Gaia Astrometry}
\label{subsec:astrometry}

\begin{figure*}[t]
	\begin{center}
		\leavevmode
		\includegraphics[width=0.95\textwidth]{f1.pdf}
	\end{center}
	\vspace{-0.7cm}
	\caption{ {\bf Reported components of NGC\,2516 in position and
    velocity space.}
    The ``core'', identified by \citet{cantatgaudin_gaia_2018} using
    Gaia DR2, is visually coincident
    with where you would think the cluster is if you looked at it through a pair
    of binoculars.
    The ``halo'' was identified by \citet{kounkel_untangling_2019}
    using a less restrictive membership assignment algorithm (discussed
    in the appendices).
    The ``field'' is a set of randomly drawn and non-overlapping stars
    within a
    $(\alpha, \delta, \pi)$ cone centered on the cluster.
		\label{fig:gaia6d}
	}
\end{figure*}

Figure~\ref{fig:gaia6d} shows the problem we would like to address.
In this figure, the ``core'' comprises \cn members that
\citet{cantatgaudin_gaia_2018} reported to have ``membership
probability'' exceeding 10\%, based on the Gaia DR2 astrometric data.
The exact meaning of this probability, and details concerning their
clustering algorithm, are discussed in Appendix~\ref{app:clustering}.
The ``halo'' comprises \cn members that
\citet{kounkel_untangling_2019} reported as members, also based on the
Gaia DR2 data.  That study did not report continuous membership
probabilities, instead opting for the binary ``member'' or ``not''.
These two different clustering methods yielded wildly different
results.  What is the true structure of \cn?
Are the core and halo truly coeval?


\section{Age-Dating the Halo of NGC\,2516}
\label{sec:agedate}

\subsection{HR Diagram from Gaia}
\label{subsec:hr}

\begin{figure*}[t]
	\begin{center}
		\leavevmode
		\subfloat{
			\includegraphics[width=0.8\textwidth]{f2a.pdf}
		}
	
		\subfloat{
			\includegraphics[width=0.8\textwidth]{f2b.pdf}
		}
	\end{center}
	\vspace{-0.7cm}
  \caption{ {\bf HR diagrams of NGC\,2516, using Gaia EDR3 photometry.}
    {\it Top:} The core (black) shows a clean sequence consistent with
    stars with a fixed age and metallicity, and varying mass.  The
    halo (blue) is similar, but somewhat noisier.  The faintest M
    dwarfs in the core and halo are brighter than in the field star
    comparison sample (gray), consistent with these stars having not
    yet reached the ZAMS.
    {\it Bottom:} Reported members of the halo, as a function of
    galactic latitude. Can the additional scatter in the halo be
    understood through differential reddening?
    {\bf Maybe}.
    \label{fig:hr}
  }
\end{figure*}

The first check on whether this membership assignment is plausible was
already performed by \citet{kounkel_untangling_2019} and more recently
by \citet{meingast_2021}.  That check is to see whether the HR
diagrams of these cluster components that were selected based on
positions and velocities photometrically support the claim that they
are coeval.

Figure~\ref{fig:hr} presents similar results to what these
investigators have already found.  The core members of the cluster
show a clean sequence consistent with stars with a fixed age and
metallicity, and varying mass.
The halo members are roughly consistent with this, but they do show
greater scatter.  One possible explanation for this scatter is that
the halo is more contaminated by field stars.
Another, explored in the other panel of the figure, is differential
reddening.
The halo is reported to span 20$^\circ$ on-sky, and varies in position from
about $b=-12^\circ$ to $b=-20^\circ$, with the stars closest to the
galactic plane also being further from the Sun
by up to 200\,pc (Figure~\ref{fig:gaia6d}).

%TODO 
Comparing these HR diagrams to the PARSEC isochrone models, we find
that X, Y, and Z.
In particular, ``the faintest M dwarfs in the core and halo are
brighter than in the field star comparison sample, consistent with
these stars having not yet reached the ZAMS.''
This is consistent with the 
main-sequence turn-off being at $Bp-Rp\approx0.05$, which implies an
age of XXX.
The resulting photometric age we adopt for the core is XXX.
For the halo, the claimed age from photometry is YYY.
Applying the same procedure to the field star comparison sample,
we get an age of ZZZ.


\subsection{Rotation from TESS}
\label{subsec:tess}

\begin{figure*}[t]
	\begin{center}
		\leavevmode
		\subfloat{
			\includegraphics[width=0.8\textwidth]{f3a.pdf}
		}
	
		\subfloat{
			\includegraphics[width=0.8\textwidth]{f3b.pdf}
		}
	\end{center}
	\vspace{-0.7cm}
	\caption{ {\bf The core and halo of \cn in the space of rotation
    period and Gaia color.}
    The {\it top} plot shows periods against a linear scale, while the
    {\it bottom}  shows them against a logarithmic scale.
		\label{fig:rot}
	}
\end{figure*}

%TODO: what are these numbers?
Duly motivated, we collected light curves from TESS photometry
for YYYY of the ZZZZ cluster members.
%TODO write the words to describe the procedure, the detrending, the
%selection function you used for these janky periods, etc.

\subsection{Lithium from Gaia-ESO and GALAH}
\label{subsec:spectra}

\section{Discussion}
\label{sec:discussion}

  \subsection{How did it form?}
  \subsection{Mass differences between center and outer reaches?}
  \subsection{Fast rotators: are we going faster than other clusters?}

\section{Conclusion}
\label{sec:conclusion}


%%%%%%%%%%%%%%%%%%%%%%%%%%%%%%%%%%%%%%%%%%%%%%%%%%%%%%%%%%%%%%%%%%%%%%%%%%%%%%%


%\clearpage
\acknowledgements
\raggedbottom

The authors thank X and Y for fruitful discussions.
%
L.G.B. and J.H. acknowledge support by the TESS GI Program, program
NUMBER, through NASA grant NUMBER.
%
This study was based in part on observations at Cerro Tololo
Inter-American Observatory at NSF's NOIRLab (NOIRLab Prop. ID
2020A-0146; 2020B-NUMBER PI: L{.}~Bouma), which is managed by the
Association of Universities for Research in Astronomy (AURA) under a
cooperative agreement with the National Science Foundation.
%
ACKNOWLEDGE PFS / CAMPANAS.
%
This paper includes data collected by the TESS mission, which are
publicly available from the Mikulski Archive for Space Telescopes
(MAST).
%
Funding for the TESS mission is provided by NASA's Science Mission
directorate.
%
We thank the TESS Architects (George Ricker, Roland Vanderspek, Dave
Latham, Sara Seager, Josh Winn, Jon Jenkins) and the many TESS team
members for their efforts to make the mission a continued success.
%

%
% The Digitized Sky Survey was produced at the Space Telescope Science
% Institute under U.S. Government grant NAG W-2166.
% Figure~\ref{fig:scene} is based on photographic data obtained using
% the Oschin Schmidt Telescope on Palomar Mountain.
%

% %
% This research made use of the NASA Exoplanet Archive, which is
% operated by the California Institute of Technology, under contract
% with the National Aeronautics and Space Administration under the
% Exoplanet Exploration Program.
% %

% Resources supporting this work were provided by the NASA High-End
% Computing (HEC) Program through the NASA Advanced Supercomputing (NAS)
% Division at Ames Research Center for the production of the SPOC data
% products.
%

% A.J.\ and R.B.\ acknowledge support from project IC120009 ``Millennium
% Institute of Astrophysics (MAS)'' of the Millenium Science Initiative,
% Chilean Ministry of Economy. A.J.\ acknowledges additional support
% from FONDECYT project 1171208.  J.I.V\ acknowledges support from
% CONICYT-PFCHA/Doctorado Nacional-21191829.  R.B.\ acknowledges support
% from FONDECYT Post-doctoral Fellowship Project 3180246.
% %
% C.T.\ and C.B\ acknowledge support from Australian Research Council
% grants LE150100087, LE160100014, LE180100165, DP170103491 and
% DP190103688.
% %
% C.Z.\ is supported by a Dunlap Fellowship at the Dunlap Institute for
% Astronomy \& Astrophysics, funded through an endowment established by
% the Dunlap family and the University of Toronto.
% %
% D.D.\ acknowledges support through the TESS Guest Investigator Program
% Grant 80NSSC19K1727.
%
%
%
% %
% Based on observations obtained at the Gemini Observatory, which is
% operated by the Association of Universities for Research in Astronomy,
% Inc., under a cooperative agreement with the NSF on behalf of the
% Gemini partnership: the National Science Foundation (United States),
% National Research Council (Canada), CONICYT (Chile), Ministerio de
% Ciencia, Tecnolog\'{i}a e Innovaci\'{o}n Productiva (Argentina),
% Minist\'{e}rio da Ci\^{e}ncia, Tecnologia e Inova\c{c}\~{a}o (Brazil),
% and Korea Astronomy and Space Science Institute (Republic of Korea).
% %
% Observations in the paper made use of the High-Resolution Imaging
% instrument Zorro at Gemini-South. Zorro was funded by the NASA
% Exoplanet Exploration Program and built at the NASA Ames Research
% Center by Steve B. Howell, Nic Scott, Elliott P. Horch, and Emmett
% Quigley.
% %
% This research has made use of the VizieR catalogue access tool, CDS,
% Strasbourg, France. The original description of the VizieR service was
% published in A\&AS 143, 23.
% %
% This work has made use of data from the European Space Agency (ESA)
% mission {\it Gaia} (\url{https://www.cosmos.esa.int/gaia}), processed
% by the {\it Gaia} Data Processing and Analysis Consortium (DPAC,
% \url{https://www.cosmos.esa.int/web/gaia/dpac/consortium}). Funding
% for the DPAC has been provided by national institutions, in particular
% the institutions participating in the {\it Gaia} Multilateral
% Agreement.
%
% (Some of) The data presented herein were obtained at the W. M. Keck
% Observatory, which is operated as a scientific partnership among the
% California Institute of Technology, the University of California and
% the National Aeronautics and Space Administration. The Observatory was
% made possible by the generous financial support of the W. M. Keck
% Foundation.
% The authors wish to recognize and acknowledge the very significant
% cultural role and reverence that the summit of Maunakea has always had
% within the indigenous Hawaiian community.  We are most fortunate to
% have the opportunity to conduct observations from this mountain.
%
% \newline
%

\software{
  %\texttt{arviz} \citep{arviz_2019},
  \texttt{astrobase} \citep{bhatti_astrobase_2018},
  %\texttt{astroplan} \citep{astroplan2018},
	%\texttt{AstroImageJ} \citep{collins_astroimagej_2017},
  \texttt{astropy} \citep{astropy_2018},
  \texttt{astroquery} \citep{astroquery_2018},
  %\texttt{BATMAN} \citep{kreidberg_batman_2015},
  %\texttt{ceres} \citep{brahm_2017_ceres},
  \texttt{cdips-pipeline} \citep{bhatti_cdips-pipeline_2019},
  \texttt{corner} \citep{corner_2016},
  %\texttt{emcee} \citep{foreman-mackey_emcee_2013},
  %\texttt{exoplanet} \citep{exoplanet:exoplanet}, and its
  %dependencies \citep{exoplanet:agol20, exoplanet:kipping13, exoplanet:luger18,
  % 	exoplanet:theano},
	%\texttt{IDL Astronomy User's Library} \citep{landsman_1995},
  \texttt{IPython} \citep{perez_2007},
	%\texttt{isochrones} \citep{morton_2015_isochrones},
	%\texttt{lightkurve} \citep{lightkurve_2018},
  \texttt{matplotlib} \citep{hunter_matplotlib_2007}, 
  %\texttt{MESA} \citep{paxton_modules_2011,paxton_modules_2013,paxton_modules_2015}
  \texttt{numpy} \citep{walt_numpy_2011}, 
  \texttt{pandas} \citep{mckinney-proc-scipy-2010},
  %\texttt{pyGAM} \citep{serven_pygam_2018_1476122},
  %\texttt{PyMC3} \citep{salvatier_2016_PyMC3},
  %\texttt{radvel} \citep{fulton_radvel_2018},
  %\texttt{scikit-learn} \citep{scikit-learn},
  \texttt{scipy} \citep{jones_scipy_2001},
  %\texttt{tesscut} \citep{brasseur_astrocut_2019},
	%\texttt{VESPA} \citep{morton_efficient_2012,vespa_2015},
  %\texttt{webplotdigitzer} \citep{rohatgi_2019},
  \texttt{wotan} \citep{hippke_wotan_2019}.
}
\ 

\facilities{
 	{\it Astrometry}:
 	Gaia \citep{gaia_collaboration_gaia_2016,gaia_collaboration_gaia_2018}.
 	{\it Imaging}:
    Second Generation Digitized Sky Survey,
    SOAR~(HRCam; \citealt{tokovinin_ten_2018}).
 	%Keck:II~(NIRC2; \url{www2.keck.hawaii.edu/inst/nirc2}).
 	%Gemini:South~(Zorro; \citealt{scott_nessi_2018}.
 	{\it Spectroscopy}:
	CTIO1.5$\,$m~(CHIRON; \citealt{tokovinin_chironfiber_2013}),
  %PFS ({\bf CITE}),
  %  MPG2.2$\,$m~(FEROS; \citealt{kaufer_commissioning_1999}),
	%AAT~(Veloce; \citealt{gilbert_veloce_2018}).
 	%Keck:I~(HIRES; \citealt{vogt_hires_1994}).
 	%{\bf VLT (number), UVES and GIRAFFE} (CITE: Pasquini et al 2002)
% 	Euler1.2m~(CORALIE),
% 	ESO:3.6m~(HARPS; \citealt{mayor_setting_2003}).
 	{\it Photometry}:
%	  ASTEP:0.40$\,$m (ASTEP400),
% 	CTIO:1.0m (Y4KCam),
% 	Danish 1.54m Telescope,
%	  El Sauce:0.356$\,$m,
% 	Elizabeth 1.0m at SAAO,
% 	Euler1.2m (EulerCam),
% 	Magellan:Baade (MagIC),
% 	Max Planck:2.2m	(GROND; \citealt{greiner_grond7-channel_2008})
% 	NTT,
% 	SOAR (SOI),
 	TESS \citep{ricker_transiting_2015}.
% 	TRAPPIST \citep{jehin_trappist_2011},
% 	VLT:Antu (FORS2).
}

% \input{TOI837_phot_table.tex}
% \input{TOI837_rv_table.tex}
% \input{ic2602_ages.tex}
% \input{starparams.tex}
% \begin{table*}
\scriptsize
\setlength{\tabcolsep}{2pt}
\centering
\caption{Literature and Measured Properties for TOI$\,$1937B}
\label{tab:compparams}
%\tablenum{2}
\begin{tabular}{llcc}
  \hline
  \hline
Other identifiers\dotfill & \\
\multicolumn{3}{c}{TIC 766593811} \\
\multicolumn{3}{c}{GAIADR2 5489726768531118848} \\
\multicolumn{3}{c}{GAIAEDR3 5489726768531118848} \\
\hline
\hline
Parameter & Description & Value & Source\\
\hline 
$\alpha_{J2016.0}$\dotfill	&Right Ascension (deg)\dotfill & 116.3706 $\pm$ 0.0098& 1	\\
$\delta_{J2016.0}$\dotfill	&Declination (deg)\dotfill & -52.3826 $\pm$ 0.0753  & 1	\\
% $l_{J2015.5}$\dotfill	&Galactic Longitude (deg)\dotfill & 265.3082 & 1	\\
% $b_{J2015.5}$\dotfill	&Galactic Latitude (deg)\dotfill & -13.5487 & 1	\\
%\\
%$NUV$\dotfill           & GALEX $NUV$ mag.\dotfill & 13.804 $\pm$ 0.004 & 2 \\
%$FUV$\dotfill           & GALEX $FUV$ mag.\dotfill & 18.466 $\pm$ 0.056 & 2 \\
\\
%B\dotfill			&Johnson B mag.\dotfill & 11.119 $\pm$ 0.107		& 2	\\
%V\dotfill			&Johnson V mag.\dotfill & 13.18 $\pm$ 0.10		& 2	\\
%$B$\tablenote{The uncertainties of the photometry have a systematic error floor applied. Even still, the global fit requires a significant scaling of the uncertainties quoted here to be consistent with our model, suggesting they are still significantly underestimated for one or more of the broad band magnitudes}\dotfill		& APASS Johnson $B$ mag.\dotfill	& 13.001 $\pm$	0.02& 2	\\
%$V$\dotfill		& APASS Johnson $V$ mag.\dotfill	& 11.808 $\pm$	0.02& 2	\\
%\\
${\rm G}$\dotfill     & Gaia $G$ mag.\dotfill     & 17.653$\pm$0.003 & 1\\
${\rm Bp}$\dotfill     & Gaia $Bp$ mag.\dotfill     & 17.950 $\pm$0.098 & 1\\
${\rm Rp}$\dotfill     & Gaia $Rp$ mag.\dotfill     & 16.246 $\pm$0.015 & 1\\
${\rm T}$\dotfill     & TESS mag.\dotfill     & 16.86$\pm$0.08 & 2\\
$\Delta I_{\rm C}$\dotfill     & SOAR Cousins-I mag diff.\dotfill & 4.3$\pm$0.X & 2\\
%$u'$\dotfill        & Sloan $u'$ mag.\dotfill & 14.706 $\pm$ 0.006& 3\\
%$g'$\dotfill		& APASS Sloan $g'$ mag.\dotfill	& 12.407 $\pm$ 0.02	&  2	\\
%$r'$\dotfill		& APASS Sloan $r'$ mag.\dotfill	& 11.311 $\pm$ 0.02	&  2	\\
%$i'$\dotfill		& APASS Sloan $i'$ mag.\dotfill	& 10.927 $\pm$ 0.04 &  2	\\
%\\
% J\dotfill			& 2MASS J mag.\dotfill & 11.717  $\pm$ 0.020	& 3	\\
% H\dotfill			& 2MASS H mag.\dotfill & 11.324 $\pm$ 0.026	    &  3	\\
% K$_{\rm S}$\dotfill			& 2MASS ${\rm K_S}$ mag.\dotfill & 11.226 $\pm$ 0.021 &  3	\\
% %\\
% W1\dotfill		& WISE1 mag.\dotfill & 11.135 $\pm$ 0.023 & 4	\\
% W2\dotfill		& WISE2 mag.\dotfill & 11.155 $\pm$ 0.020 &  4 \\
% W3\dotfill		& WISE3 mag.\dotfill & 11.160 $\pm$ 0.086& 4	\\
% W4\dotfill		& WISE4 mag.\dotfill & 9.246 $\pm$ N/A &  4	\\
\\
$\pi$\dotfill & Gaia EDR3 parallax (mas) \dotfill & 2.351 $\pm$ 0.089 &  1 \\
$d$\dotfill & Distance (pc)\dotfill & $425.3 \pm 16.1$ & 1 \\
$\mu_{\alpha'}$\dotfill		& Gaia EDR3 proper motion\dotfill & -5.387 $\pm$ 0.104 & 1 \\
                    & \hspace{3pt} in RA (mas yr$^{-1}$)	&  \\
$\mu_{\delta}$\dotfill		& Gaia EDR3 proper motion\dotfill 	&  11.349 $\pm$ 0.096 &  1 \\
                    & \hspace{3pt} in DEC (mas yr$^{-1}$) &  \\
RUWE\dotfill		& Gaia EDR3 renormalized\dotfill 	&  1.120 &  1 \\
                    & \hspace{3pt} unit weight error &  \\
% RV\dotfill & Gaia EDR3 systemic \hspace{9pt}\dotfill  & $17.44 \pm 0.64$$^{\dagger}$ & 1 \\
%                     & \hspace{3pt} radial velocity (\kms)  & \\
% RV\dotfill & Adopted systemic \hspace{9pt}\dotfill  & $17.44 \pm 0.64$$^{\dagger}$ & 1 \\
%                     & \hspace{3pt} radial velocity (\kms)  & \\
%
% \\
% $v\sin{i_\star}$\dotfill &  Rotational velocity (\kms) \hspace{9pt}\dotfill &  -- $\pm$ -- & 5 \\
% $v_{\rm mac}$\dotfill &  Macroturbulence velocity (\kms) \hspace{9pt}\dotfill &  -- $\pm$ -- & 5 \\
${\rm [Fe/H]}$\dotfill &   Metallicity \hspace{9pt}\dotfill & -- $\pm$ -- & 5 \\
$T_{\rm eff}$\dotfill &  Effective Temperature (K) \hspace{9pt}\dotfill & ---- $\pm$ --- &  6  \\
$\log{g_{\star}}$\dotfill &  Surface Gravity (cgs)\hspace{9pt}\dotfill &  x.xxx $\pm$ x.xx  &  6 \\
%
Li EW\dotfill & 6708\AA\ Equiv{.} Width (m\AA) \dotfill & NaN  & 7 \\
%
$P_{\rm rot}$\dotfill & Rotation period (d)\dotfill & NaN  & 8 \\
Age & Adopted stellar age (Myr)\dotfill & ---  &  9 \\
% $E(B-V)$\dotfill & Reddening (mag)\dotfill & $0.06 \pm 0.02$ & 9 \\
%
Spec. Type\dotfill & Spectral Type\dotfill & 	M1V{\bf FIX} & 5 \\
%
$R_\star$\dotfill & Stellar radius ($R_\odot$)\dotfill & 0.XXX$\pm$X.XXX & 6 \\
$M_\star$\dotfill & Stellar mass ($R_\odot$)\dotfill & 0.XXX$\pm$X.XXX & 6 \\
%$F_{\rm bol}$\dotfill & Stellar bolometric flux (cgs)\dotfill & (1.967$\pm$0.046)$\times10^{-9}$ & 9 \\
$A_{\rm V}$\dotfill & Interstellar reddening (mag)\dotfill & 0.XX$\pm$0.XX & 10 \\
% $U^{*}$\dotfill & Space Velocity (\kms)\dotfill & $26.24 \pm 0.46$  & \S\ref{sec:uvw} \\
% $V$\dotfill       & Space Velocity (\kms)\dotfill & $-71.52 \pm 1.68$ & \S\ref{sec:uvw} \\
% $W$\dotfill       & Space Velocity (\kms)\dotfill & $ -1.31 \pm 0.27$ & \S\ref{sec:uvw} \\
\hline
\end{tabular}
\begin{flushleft}
 \footnotesize{ \textsc{NOTE}---
% $\dagger$ Systemic RV uncertainty is the standard deviation of single-transit radial velocities, as quoted in Gaia DR2. %$*$ $U$ is in the direction of the Galactic center. \\
  {\bf FIXME}
Provenances are:
$^1$\citet{gaia_collaboration_gaia_2018},
$^2$\citet{stassun_TIC8_2019},
$^3$\citet{skrutskie_tmass_2006},
$^4$\citet{wright_WISE_2010},
$^5$CHIRON spectra,
$^6$Method~2 (cluster isochrone, Section~\ref{subsec:starparams}),
$^7$FEROS spectra,
$^8$TESS light curve,
$^9$IC~2602 ages from isochrone \& lithium depletion analyses (Section~\ref{subsec:clusterchar}),
$^{10}$Method~1 (photometric SED fit, Section~\ref{subsec:starparams}).}
\end{flushleft}
\vspace{-0.5cm}
\end{table*}

% \input{model_posterior_table.tex}

\clearpage
\bibliographystyle{yahapj}                            
\bibliography{bibliography} 

\appendix
\section{Clustering method details}
\label{app:clustering}

%
% TODO: the following two paragraphs, when quoted, are ENTIRELY FROM
% the relevant papers. they should probably be nixed in favor of
% concision.
%
\citet{cantatgaudin_gaia_2018} applied a procedure that ultimately
yielded what we call ``the core''.  Their procedure was to first query
a Gaia DR2 cone around the previously reported RA and dec of the
cluster, and within $\pm 0.5$\,mas of its previously reported
distance. 
% TODO: which was it?
The outer radius of their cone was either r2 from MWSC (CITE
Kharchenko 2013), or twice the ``cluster radius'' listed by Dias.  No
proper motion cut was applied.  They then applied an unsupervised
classification scheme called UPMASK to $G<18$\,mag stars within this
cone (CITE).  The steps of UPMASK are first to perform a k-means
clustering in the ``astrometric space'' ($\mu_{\alpha'}, \mu_\delta,
\pi$).  Then, a ``veto'' step is applied to assess whether the groups
of stars output from the k-means clustering are or are not more
concentrated than a random distribution.  This is implemented by
``comparing the total branch length of the minimum spanning tree
connecting the stars with the expected branch length for a random
uniform distribution covering the investigated field of view''.  ``To
turn this yes/no flag to a membership probability,
\citet{cantatgaudin_gaia_2018} then redraw new values of
($\mu_{\alpha'}, \mu_\delta, \pi$) for each source based on the listed
value, uncertainty, and covariance.  After a certain number of
redrawings, the final probability is the frequency with which a given
star passes the veto".  In the case of \cn, this yielded a reported
``\texttt{r50}'' within which half of the cluster members were found
to be within 0.496$^\circ$.

\citet{kounkel_untangling_2019} applied a different unsupervised
clustering method to the 5-dimensional Gaia DR2 data (omitting radial
velocities, due to their sparsity).  Their selection function (see
their Section~2) yielded $\approx 2\times 10^7$ stars, mostly within
$\approx 1$\,kpc and typically with $G<18$\,mag.  Their clustering
algorithm was the ``hierarchical density-based spatial clustering of
applications with noise'' (HDBSCAN, CITE McInnes17).  The classical
DBSCAN algorithm ``identifies clusters as overdensities in a
multi-dimensional space in which the number of sources exceeds the
required minimum number of points within a neighborhood of a
particular linking length $\epsilon$.  HDBSCAN does not depend on
$\epsilon$; instead it condenses the minimum spanning tree by pruning
off the nodes that do not meet the minimum number of sources in a
cluster and reanalyzing the nodes that do. Depending on the chosen
algorithm, it would then either find the most persistent structure
(through the excess of mass method), or return clusters as the leaves
of the tree (which results in somewhat more homogeneous clusters). In
both cases it is more effective at finding structures of varying
densities in a given data set than DBSCAN.''
`` The two main parameters that control HDBSCAN are the number of
sources in a cluster and the number of samples. The former is the
parameter that rejects groupings that are too small; the latter sets
the threshold of how conservative the algorithm is in its
considerations of the background noise (even if the resulting noisy
groupings do meet the minimum cluster size).  By default, the sample
size is set to the same value as the cluster size, but it is possible
to adjust them separately.''


\listofchanges

%\allauthors
\end{document}
