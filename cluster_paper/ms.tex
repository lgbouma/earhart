%%%%%%%%%%%%%%%%%%%%%%%%%%%%%%%%%%%%%%%%%%%%%%%%%%%%%%%%%%%%%%%%%%%%%%%%%%%%%%%

\documentclass[12pt,twocolumn,tighten]{aastex63}
%\documentclass[12pt,twocolumn,tighten,trackchanges]{aastex63}
\usepackage{amsmath,amstext,amssymb}
\usepackage[T1]{fontenc}
\usepackage{apjfonts}
\usepackage[figure,figure*]{hypcap}
\usepackage{graphics,graphicx}
\usepackage{hyperref}
\usepackage{natbib}
\usepackage[caption=false]{subfig} % for subfloat
\usepackage{enumitem} % for specific spacing of enumerate
\usepackage{epigraph}

\renewcommand*{\sectionautorefname}{Section} %for \autoref
\renewcommand*{\subsectionautorefname}{Section} %for \autoref

\newcommand{\cn}{NGC\,2516} % cluster name

%TODO:
%
% n rotation period logs: 2238 % ~/proj/cdips/results/allvariability_reports/NGC_2516/logs
% n all variability reports: 2201 % ~/proj/cdips/results/allvariability_reports/NGC_2516/reports

\newcommand{\numcdipslcs}{671{,}901\ } % drivers.yield_calculation.count_cdips_lcs.py; 2021.01.13
\newcommand{\numsouthernuniqlcs}{483{,}407\ } % drivers.yield_calculation.count_cdips_lcs.py; 2021.01.13



\newcommand{\kms}{\,km\,s$^{-1}$}

%% Reintroduced the \received and \accepted commands from AASTeX v5.2.
%% Add "Submitted to " argument.
\received{---}
\revised{---}
\accepted{---}
\submitjournal{ApJL.}
\shorttitle{Core \& Halo of NGC\,2516}

\begin{document}

\defcitealias{bouma_wasp4b_2019}{B19}
\defcitealias{cantatgaudin_gaia_2018}{CG18}
\defcitealias{kounkel_untangling_2019}{KC19}

\title{
  Cluster Difference Imaging Photometric Survey. III.
  The Coeval Halo and Core of NGC\,2516
}

%\suppressAffiliations
%\NewPageAfterKeywords
\correspondingauthor{L.\,G.\,Bouma}
\email{luke@astro.princeton.edu}

\author[0000-0002-0514-5538]{L. G. Bouma}
\affiliation{Department of Astrophysical Sciences, Princeton University, 4 Ivy Lane, Princeton, NJ 08540, USA}

\author[0000-0001-8732-6166]{J. D. Hartman}
\affiliation{Department of Astrophysical Sciences, Princeton University, 4 Ivy Lane, Princeton, NJ 08540, USA}

\author[0000-0002-4265-047X]{J. N. Winn}
\affiliation{Department of Astrophysical Sciences, Princeton University, 4 Ivy Lane, Princeton, NJ 08540, USA}

\author[0000-0001-7204-6727]{G. \'A. Bakos}
\affiliation{Department of Astrophysical Sciences, Princeton University, 4 Ivy Lane, Princeton, NJ 08540, USA}



% Context, aims, methods, results, conclusions.
\begin{abstract}
  Recent analyses of the Gaia data have reported the existence of
  diffuse stellar populations (``halos'') surrounding nearby open clusters.  The stars in these
  halos could have escaped from the cores of the cluster, or they
  could be remnants of a larger and lower-density star formation
  complex that has since dispersed.  They could also be field
  stars---false positives given by overly aggressive clustering algorithms.
  In this study, we focus on a single reported 250\,pc-long halo
  around the $\approx$120\,Myr open cluster NGC\,2516.  The classical tidal radius of
  this cluster is $\sim$XX\,pc.  Combining photometry from Gaia,
  rotation periods from TESS, and lithium measurements from
  the Gaia-ESO and GALAH surveys, we find that: i) the halo of
  NGC\,2516 is real and coeval with the core; ii) at fixed stellar
  mass and age, rapidly rotating stars have greater lithium abundances
  than slowly rotating stars; and iii) given Gaia-selected halo
  members, the contamination rate of field stars in the halo (XX\%) is
  larger than that in the core (YY\%).  This work expands the set of
  confirmed NGC\,2516 members by a factor of $\approx$2, and
  quantifies the degree to which Gaia-based analyses can currently
  identify dispersed stellar populations against a background of field
  star contaminants.  Implications for spectroscopic survey targeting, open cluster dispersal, and planet searches around young
  stars are discussed.
\end{abstract}

\keywords{
	stellar ages (1581),
  stellar associations (1582),
  open clusters (1160),
  %stellar dynamics (1596),
  stellar rotation (1629)
}

%%%%%%%%%%%%%%%%%%%%%%%%%%%%%%%%%%%%%%%%%%%%%%%%%%%%%%%%%%%%%%%%%%%%%%%%%%%%%%%


\section{Introduction}

In the traditional picture of star formation molecular clouds go
gravitationally unstable, and collapse into little knots.  The knots
produce groups of stars close to each other.  The resulting ``open star
clusters'' are, as with any first-order guess, spherical.

How does the cluster evolve?  The answer depends on the total stellar
mass.  The smallest 90\% of star clusters disperse during the ``embedded
phase'' (CITE Lada \& Lada 03, Krumholz 19 ARAA).  This is driven mostly
by {\it process A} (CITE).  The more massive clusters make it maybe 100
Myr. Maybe a bit longer.  Their evaporation is thought to be driven by
collisions with molecular clouds (CITE, Spitzer 1958, maybe Sec 7.2 of
Ryden), and the galactic tide (CITE), and {\it process A}, and {\it
process B}.  Of course, the expected evaporation time should depend on
factors including the mass of the cluster itself, and how many high-mass
(O and B) stars form, since their winds [and maybe supernovae] clear out
most of the cloud, and the initial density of the gas cloud to begin
with.

Assuming that the cluster achieves virial equilibrium, stellar fly-bys
then conspire to segregate the stellar mass distribution within the
cluster, evaporating the lowest mass stars soonest (2T+U=0, U=-GM/r,
assuming equipartition of the specific ``thermal energy'' per star, then
kT = 0.5$mv^2$ at lower masses requires higher velocities).  

Identification of the stars that disperse into the galactic field is an
important task for understanding the conditions under which stars and
star clusters form, and for understanding how they subsequently evolve.
For instance, how does the process of radial migration across the
galactic disk affect cluster dispersal?  Did the Sun form in an open
cluster? If so, how massive was its cluster, and is there any hope at
identifying the stars that formed near the Sun?  Qualitatively, the
dispersal of open clusters also provides perhaps the best test case for
the concept of ``chemical tagging'', aka.\ ``galactic archaeology''
(CITE).  If you can't chemically tag stars that are kinematically
associated, why would you be able to do so once they've dispersed
throughout the galactic disk?

Outside of the issue of star formation, identifying the remnant halos of
open clusters is important for a separate project: that of discovering
young transiting planets.  Young transiting planets are hard to find
because young stars are rare (CITE), and often reside in crowded regions
of the sky (CITE).  If the halos of nearby star clusters can be reliably
identified, this could expand the census of nearby sub-Gyr stars by a
factor of 2 or 3 (CITE).  A fortuitous benefit of searching for planets
in cluster environments is that issues with stellar crowding are also
alleviated.

Recent clustering studies using Gaia have begun to report the
identification of structures that could correspond to low-density halos
of stars that may have evaporated from open clusters (e.g., Kounkel+19,
Kounkel+20, Meingast+21).  However, different clustering methods on the
Gaia data tend to give different results (CITE: Hunt \& Reffert 2020).
Employing say Gaussian Mixture Modelling, or any analogous method that
requires ``clusters'' to be ellipses in phase-space (position or
velocity) unsurpsingly yields open clusters that are roughly elliptical.
Some unsupervised clustering methods lead to comparable ``roughly
elliptical'' results \citep[][hereafter
\citetalias{cantatgaudin_gaia_2018}]{cantatgaudin_gaia_2018}.  Other
unsupervised methods such as HDBScan (e.g., \citep[][hereafter
\citetalias{kounkel_untangling_2019}]{kounkel_untangling_2019}) have
been found to yield additional structures, particularly in lower density
regions such as the Psc-Eri stream (CITE: Meingast+18, Curtis+19,
Newton+21), but also more generally around many open clusters
\citep{kounkel_untangling_2019}.  Unsupervised approaches that
incorporate physical constraints like imposing a maximum velocity
dispersion on putative members yield similar results, potentially with
higher purity (Meingast+21).

% TODO: make the bonafide number below quantitative
We've recently been making TESS light curves of age-dated stars across
the sky, as part of a Cluster Difference Imaging Photometry Survey
(CDIPS, CITE Bouma+19).  Our analysis of Cycle 1 (Sectors 1-13) yielded
light curves of \numsouthernuniqlcs candidate cluster members in the
Southern Ecliptic hemisphere, available on
MAST\footnote{\url{https://archive.stsci.edu/hlsp/cdips}}.  Based on
rotation periods, $\approx$25\% appear to be bonafide cluster members. 

As part of a broader project of identifying a large and clean sample of
young stars for a transit search, we focus in his paper on a rather
modest question: in just a single rich southern open cluster, is the
cluster halo coeval with the core?  The cluster we chose for this
analysis was NGC\,2516, since it was young ($\sim$120\,Myr) and close
($d\approx400$\,pc) enough to facilitate rotation measurements using
TESS, and some degree of spectroscopic analysis.  We want to know: is
the halo real? To what extent can we use Gaia alone to reliably identify
age-dated needles in the haystack of boring field stars?  And more
generally, what are the implications for the evolution of open clusters
if they do have halos?

Section~\ref{sec:gaia} presents the astrometric data from Gaia, and
clarifies our usage of the terms ``core'' and ``halo''.
Section~\ref{sec:agedate} age-dates the halo of NGC\,2516, based on Gaia
photometry (Section~\ref{subsec:hr}), TESS gyrochronology
(Section~\ref{subsec:tess}), and lithium depletion
(Section~\ref{subsec:lithium}).  In Section~\ref{sec:discussion} we
discuss the implications of this analysis for NGC\,2516 specifically and
stellar spin-down and open cluster evolution generally.
Section~\ref{sec:conclusion} gives our conclusions.


\section{A 250\,pc Halo around a Core?}
\label{sec:gaia}
%
% What is the core? What is the halo?
%

\subsection{Gaia Astrometry}
\label{subsec:astrometry}

\begin{figure*}[t]
	\begin{center}
		\leavevmode
		\includegraphics[width=0.95\textwidth]{f1.pdf}
	\end{center}
	\vspace{-0.7cm}
	\caption{ {\bf Reported components of NGC\,2516 in position and
    velocity space.}
    The ``core'', identified by \citet{cantatgaudin_gaia_2018} using
    Gaia DR2, is visually coincident
    with where you would think the cluster is if you looked at it through a pair
    of binoculars.
    The ``halo'' was identified by \citet{kounkel_untangling_2019}
    using a less restrictive membership assignment algorithm (discussed
    in the appendices).
    The ``field'' is a set of randomly drawn and non-overlapping stars
    within a
    $(\alpha, \delta, \pi)$ cone centered on the cluster.
		\label{fig:gaia6d}
	}
\end{figure*}

We chose the stars to focus on based on previously reported
memberships of NGC\,2516, available in the literature.  While a number
of pre-Gaia catalogs were available, the purity and accuracy of the
Gaia-derived results are expected to be the current state of the art.
We therefore adopted what we viewed as the most interesting clustering
samples to compare: those of \citet{cantatgaudin_gaia_2018}
(\citetalias{cantatgaudin_gaia_2018}) and
\citet{kounkel_untangling_2019}
(\citetalias{kounkel_untangling_2019})\footnote{A third sample was
also recently reported by \citet{meingast_2021}. Although the results
of this study were made publicly available too late to be included in
our analysis,  1577 of the 1860 \citet{meingast_2021} sources (85\%)
were included in the \citet{kounkel_untangling_2019} sample.  A
complete set of \cn members could very well include some of the newly
reported candidate members from \citet{meingast_2021}.}. While we
could have performed our own clustering analysis based on the Gaia
data, such an effort would be replicating the work of these
investigators. And why would our clustering methods be better than
theirs? We opt instead to use their studies as starting points.

% Theia 613
\citetalias{cantatgaudin_gaia_2018} reported 1106 candidate members of
\cn, brighter than $G=18$\,mag.  \citet{kounkel_untangling_2019}
reported 3003 candidate members, and included stars up to
$\approx$1\,mag fainter.  Each study used unsupervised clustering
based on the second Gaia Data Release, detailed in
Appendix~\ref{app:clustering}.

%TODO calculate these physical sizes carefully
Figure~\ref{fig:gaia6d} shows the cluster members reported by each
study.  The \citetalias{cantatgaudin_gaia_2018} members are all within
a few degrees of the cluster center, while the
\citetalias{kounkel_untangling_2019} members span tens of degrees.  In
physical units, this corresponds to a size difference of XX\,pc versus
YY\,pc.  This difference is somewhat tautological, because
\citetalias{cantatgaudin_gaia_2018} did not extend their search for
members out to tens of degrees from the cluster center, since their
clustering algorithm relied on the contrast between cluster and field
stars in the astrometric space of ($\mu_{\alpha'}, \mu_\delta, \pi$).
Nonetheless, the membership catalog of \citetalias{cantatgaudin_gaia_2018}
echoes that of many previous investigators (CITE, CITE, CITE), and is
consistent with the general {\it visual} impression that one has when
looking at \cn visually: it seems to be at most {\bf X$^\circ$}, and
not a big stream.
This discrepancy begs the question.  What is the true structure of
\cn?  Are the core and halo truly coeval?

To answer these questions, we adopt the
\citetalias{cantatgaudin_gaia_2018} stars as candidate ``core''
members and the \citetalias{kounkel_untangling_2019} stars as
candidate ``halo'' members.  If a star is in both lists, it is a
``core'' member.  1091 of the 1106 core members are in both lists.
This leaves 1912 candidate members in the halo.
Concatenating the two lists yields 3018 candidate members.

In this work, we also define a set of nearby field stars in the
``neighborhood'' of \cn.  Based on the observed distribution of halo
members, we draw these stars randomly from the following ``cube'' in
right ascension, declination, and parallax:
\begin{align}
  $\alpha$\,[{\rm deg}] &\in [108, 132], \\
  $\delta$\,[{\rm deg}] &\in [-76, -45], \\
  $\pi$\,[{\rm mas}] &\in [1.5, 4.0].
\end{align}
We imposed a magnitude limit of $G=19$\,mag, and ran the queries
using the \texttt{astroquery.gaia} module (CITE).
% TODO: why?
We allowed the number of stars in the comparison sample to exceed that
in the cluster sample by a factor of a few, to ensure broad sampling
of stellar masses and evolutionary states.
We also required the comparison sample to not overlap with the cluster
sample, which led to the omission of 1.1\% of the stars drawn over the
volume noted above.


\section{Age-Dating the Halo of NGC\,2516}
\label{sec:agedate}

\subsection{HR Diagram from Gaia}
\label{subsec:hr}

\begin{figure*}[t]
	\begin{center}
		\leavevmode
		\subfloat{
			\includegraphics[width=0.75\textwidth]{f2a.pdf}
		}
	
		\subfloat{
			\includegraphics[width=0.75\textwidth]{f2b.pdf}
		}
	\end{center}
	\vspace{-0.7cm}
  \caption{ {\bf HR diagrams of NGC\,2516, using Gaia EDR3 photometry.}
    {\it Top:} The core (black) shows a clean sequence consistent with
    stars with a fixed age and metallicity, and varying mass.  The
    halo (blue) is similar, but somewhat noisier.  The faintest M
    dwarfs in the core and halo are brighter than in the field star
    comparison sample (gray), consistent with these stars having not
    yet reached the ZAMS.
    {\it Bottom:} Reported members of the halo, as a function of
    galactic latitude. Can the additional scatter in the halo be
    understood through differential reddening?
    {\bf Maybe}.
    \label{fig:hr}
  }
\end{figure*}

The first check on whether this membership assignment is plausible was
already performed by \citetalias{cantatgaudin_gaia_2018},
\citetalias{kounkel_untangling_2019}, and more recently by
\citet{meingast_2021}.  That check is to see whether the HR diagrams
of the cluster members support the claim that they are coeval.

Figure~\ref{fig:hr} presents similar results to what these
investigators have already found.  The core members of the cluster
show a clean sequence consistent with stars with a fixed age and
metallicity, and varying mass.
The halo members are roughly consistent with this, but they do show
greater scatter.  One possible explanation for this scatter is that
the halo is more contaminated by field stars.
Another, explored in the other panel of the figure, is differential
reddening.
The halo is reported to span 20$^\circ$ on-sky, and varies in position from
about $b=-12^\circ$ to $b=-20^\circ$, with the stars closest to the
galactic plane also being further from the Sun
by up to 200\,pc (Figure~\ref{fig:gaia6d}).

%TODO 
Comparing these HR diagrams to the PARSEC isochrone models, we find
that X, Y, and Z.
In particular, ``the faintest M dwarfs in the core and halo are
brighter than in the field star comparison sample, consistent with
these stars having not yet reached the ZAMS.''
This is consistent with the 
main-sequence turn-off being at $Bp-Rp\approx0.05$, which implies an
age of XXX.
The resulting photometric age we calculate for the core is XXX.
For the halo, the claimed age from photometry is YYY.
Applying the same procedure to the field star comparison sample,
we get an age of ZZZ.


\subsection{Rotation from TESS}
\label{subsec:tess}

\begin{figure*}[t]
	\begin{center}
		\leavevmode
		\subfloat{
			\includegraphics[width=0.75\textwidth]{f3a.pdf}
		}
	
		\subfloat{
			\includegraphics[width=0.75\textwidth]{f3b.pdf}
		}
	\end{center}
	\vspace{-0.7cm}
	\caption{ {\bf The core and halo of \cn in the space of rotation
    period and Gaia color.}
    The {\it top} plot shows periods against a linear scale, while the
    {\it bottom}  shows them against a logarithmic scale.
		\label{fig:rot}
	}
\end{figure*}



%TODO: what are these numbers?
Duly motivated, we collected light curves from TESS photometry
for YYYY of the ZZZZ cluster members.
%TODO write the words to describe the procedure, the detrending, the
%selection function you used for these janky periods, etc.

First, we collected the XXXX Gaia DR2 source\_ids corresponding to the
core and halo members.  For each source, we first retrieved all
available CDIPS light curves, on a per-sector basis.  Then, we
detrended the systematics in each sector individually, and stitched
together the resulting light curves before searching for the
periodicity.  Some details regarding our detrending approach are
discussed in Appendix~\ref{app:detrending}.  After applying a
detrending step aimed at removing systematic trends, we proceeding
with a few small cleaning steps aimed at improving the purity of the
rotation period measurements: we masked 0.7 days at the beginning and
end of each spacecraft orbit, and ran a sliding standard-deviation
rejection window over the light curve, which removed any outlying
points within $\pm3\times$MAD of the median in each window.

We then measured the rotation period of the resulting light curve.  We
used the the aperture radius that, based on theoretical expectations,
was expected to give the optimal balance between light from the target
and background-light (CITE Sullivan15).  This typically resulted in an
aperture radius of either 1 or 1.5 pixels.  To measure the periods, we
used the periodogram implementations in \texttt{astrobase}, in
particular the Stellingwerf PDM periodogram (CITE), along with the
more traditional Lomb-Scargle (CITE).  We recorded the top five
periodogram peaks from each method, and their corresponding powers.
Finally, as a check on crowding, we also recorded the number of stars
within the aperture of equal brightness, and of brightness with 1.25
and 2.5 TESS magnitudes of the target star.

Figure~\ref{fig:rot} shows the resulting rotation periods.  The points
on this plot are light curves for which the peak Lomb Scargle
periodogram period was below 15 days, had power exceeding 0.08, and
had at most one equal-brightness companion within the aperture.  These
selection criteria are entirely heuristic.  
But splitting the sample into the ``core'' and the ``halo'', they
suggest that the halo is real.

The resulting gyrochronology age we find for the core is XXX.
For the halo, the claimed age from gyrochronology is YYY.
Applying the same procedure to the field star comparison sample,
we get an age of ZZZ.

\subsubsection{Kinematics $\otimes$ Rotation}

\begin{figure*}[t]
	\begin{center}
		\leavevmode
		\includegraphics[width=0.95\textwidth]{f4.pdf}
	\end{center}
	\vspace{-0.7cm}
	\caption{ {\bf Gaia-based components of NGC\,2516 in position and
    velocity space, cross-matched against the rotators.}
		\label{fig:gaia6d_x_rotn}
	}
\end{figure*}

How far away from the core, in position and velocity space, does the
halo actually extend?  We can explore this by crossmatching the
rotator sample against the Gaia data.  Figure~\ref{fig:gaia6d_x_rotn}
shows the result. 



\subsection{Lithium from Gaia-ESO and GALAH}
\label{subsec:lithium}

\begin{figure*}[t]
	\begin{center}
		\leavevmode
		\subfloat{
			\includegraphics[width=0.75\textwidth]{f5a.pdf}
		}
	
		\subfloat{
			\includegraphics[width=0.75\textwidth]{f5b.pdf}
		}
	\end{center}
	\vspace{-0.7cm}
  \caption{ {\bf Lithium equivalent widths in NGC\,2516,}
  plotted against extinction-corrected color ({\it top}), and the
  combination of rotation period and extinction-corrected color ({\it
  bottom}).
  \label{fig:lithium}
  }
\end{figure*}

Figure~\ref{fig:lithium} shows the results.  At fixed stellar mass
(and age), it seems that the rapid rotators tend to show elevated
lithium equivalent widths.  Similar trends have been previously noted
in the Pleiades by \citet{soderblom_evolution_1993} and
\citet{bouvier_pleiades_lirot_2018}, in the Psc-Eri stream by
\citet{arancibia_2020}, and in M\,35 by \citet{jeffries_m35_li_2020}.

% TODO: this paragraph more or less parrots the processes described by
% Bouvier+20. are there others? is there a better way to describe
% these?
The trend contradicts what might be the naive expectation that rapidly
rotating stars should show more vigorous convection, leading to
earlier destruction of photospheric Li.  Possible physical
explanations include processes both internal and external to the star
(CITE see the recent review by Bouvier+20).  Internal processes would
be tied to the interplay between surface rotation, differential
rotation in the interior, and the convective mixing efficiency (e.g.,
CITE Siess + Livio 1997, Baraffe+2017).  Internal processes could also
include the possibility that stronger magnetic fields in the stellar
interior inhibit convection (READ e.g., Ventura+98, Chabrier+07,
Somers + Pinsonneault 2014).  An external process that could also be
important is the effect of star-disk magnetic braking during the PMS
phase (CITE: magnetic braking).  Longer disk lifetimes would lead to
the star's outer convective zone being ``locked'' for longer while the
radiative core continues contracting.  The resulting differential
rotation and rotational mixing could drive the lithium depletion
(CITE: Bouvier 08, Eggenberger+12).



\section{Discussion}
\label{sec:discussion}

  \subsection{What is the contamination fraction in the halo? Does it change vs. location?}
  \subsection{Are the ``very slow rotators'' bonafide members?}
  Probably not. They are not isotropically distributed around the
  cluster.
  \subsection{How did the halo form?}
  \subsection{Mass differences between center and outer reaches?}
  \subsection{Fast rotators: are we going faster than other clusters?}

\section{Conclusion}
\label{sec:conclusion}


%%%%%%%%%%%%%%%%%%%%%%%%%%%%%%%%%%%%%%%%%%%%%%%%%%%%%%%%%%%%%%%%%%%%%%%%%%%%%%%


%\clearpage
\acknowledgements
\raggedbottom

The authors thank X and Y for fruitful discussions.
%
L.G.B. and J.H. acknowledge support by the TESS GI Program, program
NUMBER, through NASA grant NUMBER.
%
This study was based in part on observations at Cerro Tololo
Inter-American Observatory at NSF's NOIRLab (NOIRLab Prop. ID
2020A-0146; 2020B-NUMBER PI: L{.}~Bouma), which is managed by the
Association of Universities for Research in Astronomy (AURA) under a
cooperative agreement with the National Science Foundation.
%
ACKNOWLEDGE PFS / CAMPANAS.
%
This paper includes data collected by the TESS mission, which are
publicly available from the Mikulski Archive for Space Telescopes
(MAST).
%
Funding for the TESS mission is provided by NASA's Science Mission
directorate.
%
We thank the TESS Architects (George Ricker, Roland Vanderspek, Dave
Latham, Sara Seager, Josh Winn, Jon Jenkins) and the many TESS team
members for their efforts to make the mission a continued success.
%

%
% The Digitized Sky Survey was produced at the Space Telescope Science
% Institute under U.S. Government grant NAG W-2166.
% Figure~\ref{fig:scene} is based on photographic data obtained using
% the Oschin Schmidt Telescope on Palomar Mountain.
%

% %
% This research made use of the NASA Exoplanet Archive, which is
% operated by the California Institute of Technology, under contract
% with the National Aeronautics and Space Administration under the
% Exoplanet Exploration Program.
% %

% Resources supporting this work were provided by the NASA High-End
% Computing (HEC) Program through the NASA Advanced Supercomputing (NAS)
% Division at Ames Research Center for the production of the SPOC data
% products.
%

% A.J.\ and R.B.\ acknowledge support from project IC120009 ``Millennium
% Institute of Astrophysics (MAS)'' of the Millenium Science Initiative,
% Chilean Ministry of Economy. A.J.\ acknowledges additional support
% from FONDECYT project 1171208.  J.I.V\ acknowledges support from
% CONICYT-PFCHA/Doctorado Nacional-21191829.  R.B.\ acknowledges support
% from FONDECYT Post-doctoral Fellowship Project 3180246.
% %
% C.T.\ and C.B\ acknowledge support from Australian Research Council
% grants LE150100087, LE160100014, LE180100165, DP170103491 and
% DP190103688.
% %
% C.Z.\ is supported by a Dunlap Fellowship at the Dunlap Institute for
% Astronomy \& Astrophysics, funded through an endowment established by
% the Dunlap family and the University of Toronto.
% %
% D.D.\ acknowledges support through the TESS Guest Investigator Program
% Grant 80NSSC19K1727.
%
%
%
% %
% Based on observations obtained at the Gemini Observatory, which is
% operated by the Association of Universities for Research in Astronomy,
% Inc., under a cooperative agreement with the NSF on behalf of the
% Gemini partnership: the National Science Foundation (United States),
% National Research Council (Canada), CONICYT (Chile), Ministerio de
% Ciencia, Tecnolog\'{i}a e Innovaci\'{o}n Productiva (Argentina),
% Minist\'{e}rio da Ci\^{e}ncia, Tecnologia e Inova\c{c}\~{a}o (Brazil),
% and Korea Astronomy and Space Science Institute (Republic of Korea).
% %
% Observations in the paper made use of the High-Resolution Imaging
% instrument Zorro at Gemini-South. Zorro was funded by the NASA
% Exoplanet Exploration Program and built at the NASA Ames Research
% Center by Steve B. Howell, Nic Scott, Elliott P. Horch, and Emmett
% Quigley.
% %
% This research has made use of the VizieR catalogue access tool, CDS,
% Strasbourg, France. The original description of the VizieR service was
% published in A\&AS 143, 23.
% %
% This work has made use of data from the European Space Agency (ESA)
% mission {\it Gaia} (\url{https://www.cosmos.esa.int/gaia}), processed
% by the {\it Gaia} Data Processing and Analysis Consortium (DPAC,
% \url{https://www.cosmos.esa.int/web/gaia/dpac/consortium}). Funding
% for the DPAC has been provided by national institutions, in particular
% the institutions participating in the {\it Gaia} Multilateral
% Agreement.
%
% (Some of) The data presented herein were obtained at the W. M. Keck
% Observatory, which is operated as a scientific partnership among the
% California Institute of Technology, the University of California and
% the National Aeronautics and Space Administration. The Observatory was
% made possible by the generous financial support of the W. M. Keck
% Foundation.
% The authors wish to recognize and acknowledge the very significant
% cultural role and reverence that the summit of Maunakea has always had
% within the indigenous Hawaiian community.  We are most fortunate to
% have the opportunity to conduct observations from this mountain.
%
% \newline
%

\software{
  %\texttt{arviz} \citep{arviz_2019},
  \texttt{astrobase} \citep{bhatti_astrobase_2018},
  %\texttt{astroplan} \citep{astroplan2018},
	%\texttt{AstroImageJ} \citep{collins_astroimagej_2017},
  \texttt{astropy} \citep{astropy_2018},
  \texttt{astroquery} \citep{astroquery_2018},
  %\texttt{BATMAN} \citep{kreidberg_batman_2015},
  %\texttt{ceres} \citep{brahm_2017_ceres},
  \texttt{cdips-pipeline} \citep{bhatti_cdips-pipeline_2019},
  \texttt{corner} \citep{corner_2016},
  %\texttt{emcee} \citep{foreman-mackey_emcee_2013},
  %\texttt{exoplanet} \citep{exoplanet:exoplanet}, and its
  %dependencies \citep{exoplanet:agol20, exoplanet:kipping13, exoplanet:luger18,
  % 	exoplanet:theano},
	%\texttt{IDL Astronomy User's Library} \citep{landsman_1995},
  \texttt{IPython} \citep{perez_2007},
	%\texttt{isochrones} \citep{morton_2015_isochrones},
	%\texttt{lightkurve} \citep{lightkurve_2018},
  \texttt{matplotlib} \citep{hunter_matplotlib_2007}, 
  %\texttt{MESA} \citep{paxton_modules_2011,paxton_modules_2013,paxton_modules_2015}
  \texttt{numpy} \citep{walt_numpy_2011}, 
  \texttt{pandas} \citep{mckinney-proc-scipy-2010},
  %\texttt{pyGAM} \citep{serven_pygam_2018_1476122},
  %\texttt{PyMC3} \citep{salvatier_2016_PyMC3},
  %\texttt{radvel} \citep{fulton_radvel_2018},
  %\texttt{scikit-learn} \citep{scikit-learn},
  \texttt{scipy} \citep{jones_scipy_2001},
  %\texttt{tesscut} \citep{brasseur_astrocut_2019},
	%\texttt{VESPA} \citep{morton_efficient_2012,vespa_2015},
  %\texttt{webplotdigitzer} \citep{rohatgi_2019},
  \texttt{wotan} \citep{hippke_wotan_2019}.
}
\ 

\facilities{
 	{\it Astrometry}:
 	Gaia \citep{gaia_collaboration_gaia_2016,gaia_collaboration_gaia_2018}.
 	{\it Imaging}:
    Second Generation Digitized Sky Survey,
    SOAR~(HRCam; \citealt{tokovinin_ten_2018}).
 	%Keck:II~(NIRC2; \url{www2.keck.hawaii.edu/inst/nirc2}).
 	%Gemini:South~(Zorro; \citealt{scott_nessi_2018}.
 	{\it Spectroscopy}:
	CTIO1.5$\,$m~(CHIRON; \citealt{tokovinin_chironfiber_2013}),
  %PFS ({\bf CITE}),
  %  MPG2.2$\,$m~(FEROS; \citealt{kaufer_commissioning_1999}),
	%AAT~(Veloce; \citealt{gilbert_veloce_2018}).
 	%Keck:I~(HIRES; \citealt{vogt_hires_1994}).
 	%{\bf VLT (number), UVES and GIRAFFE} (CITE: Pasquini et al 2002)
% 	Euler1.2m~(CORALIE),
% 	ESO:3.6m~(HARPS; \citealt{mayor_setting_2003}).
 	{\it Photometry}:
%	  ASTEP:0.40$\,$m (ASTEP400),
% 	CTIO:1.0m (Y4KCam),
% 	Danish 1.54m Telescope,
%	  El Sauce:0.356$\,$m,
% 	Elizabeth 1.0m at SAAO,
% 	Euler1.2m (EulerCam),
% 	Magellan:Baade (MagIC),
% 	Max Planck:2.2m	(GROND; \citealt{greiner_grond7-channel_2008})
% 	NTT,
% 	SOAR (SOI),
 	TESS \citep{ricker_transiting_2015}.
% 	TRAPPIST \citep{jehin_trappist_2011},
% 	VLT:Antu (FORS2).
}

% \input{TOI837_phot_table.tex}
% \input{TOI837_rv_table.tex}
% \input{ic2602_ages.tex}
% \input{starparams.tex}
% \begin{table*}
\scriptsize
\setlength{\tabcolsep}{2pt}
\centering
\caption{Literature and Measured Properties for TOI$\,$1937B}
\label{tab:compparams}
%\tablenum{2}
\begin{tabular}{llcc}
  \hline
  \hline
Other identifiers\dotfill & \\
\multicolumn{3}{c}{TIC 766593811} \\
\multicolumn{3}{c}{GAIADR2 5489726768531118848} \\
\multicolumn{3}{c}{GAIAEDR3 5489726768531118848} \\
\hline
\hline
Parameter & Description & Value & Source\\
\hline 
$\alpha_{J2016.0}$\dotfill	&Right Ascension (deg)\dotfill & 116.3706 $\pm$ 0.0098& 1	\\
$\delta_{J2016.0}$\dotfill	&Declination (deg)\dotfill & -52.3826 $\pm$ 0.0753  & 1	\\
% $l_{J2015.5}$\dotfill	&Galactic Longitude (deg)\dotfill & 265.3082 & 1	\\
% $b_{J2015.5}$\dotfill	&Galactic Latitude (deg)\dotfill & -13.5487 & 1	\\
%\\
%$NUV$\dotfill           & GALEX $NUV$ mag.\dotfill & 13.804 $\pm$ 0.004 & 2 \\
%$FUV$\dotfill           & GALEX $FUV$ mag.\dotfill & 18.466 $\pm$ 0.056 & 2 \\
\\
%B\dotfill			&Johnson B mag.\dotfill & 11.119 $\pm$ 0.107		& 2	\\
%V\dotfill			&Johnson V mag.\dotfill & 13.18 $\pm$ 0.10		& 2	\\
%$B$\tablenote{The uncertainties of the photometry have a systematic error floor applied. Even still, the global fit requires a significant scaling of the uncertainties quoted here to be consistent with our model, suggesting they are still significantly underestimated for one or more of the broad band magnitudes}\dotfill		& APASS Johnson $B$ mag.\dotfill	& 13.001 $\pm$	0.02& 2	\\
%$V$\dotfill		& APASS Johnson $V$ mag.\dotfill	& 11.808 $\pm$	0.02& 2	\\
%\\
${\rm G}$\dotfill     & Gaia $G$ mag.\dotfill     & 17.653$\pm$0.003 & 1\\
${\rm Bp}$\dotfill     & Gaia $Bp$ mag.\dotfill     & 17.950 $\pm$0.098 & 1\\
${\rm Rp}$\dotfill     & Gaia $Rp$ mag.\dotfill     & 16.246 $\pm$0.015 & 1\\
${\rm T}$\dotfill     & TESS mag.\dotfill     & 16.86$\pm$0.08 & 2\\
$\Delta I_{\rm C}$\dotfill     & SOAR Cousins-I mag diff.\dotfill & 4.3$\pm$0.X & 2\\
%$u'$\dotfill        & Sloan $u'$ mag.\dotfill & 14.706 $\pm$ 0.006& 3\\
%$g'$\dotfill		& APASS Sloan $g'$ mag.\dotfill	& 12.407 $\pm$ 0.02	&  2	\\
%$r'$\dotfill		& APASS Sloan $r'$ mag.\dotfill	& 11.311 $\pm$ 0.02	&  2	\\
%$i'$\dotfill		& APASS Sloan $i'$ mag.\dotfill	& 10.927 $\pm$ 0.04 &  2	\\
%\\
% J\dotfill			& 2MASS J mag.\dotfill & 11.717  $\pm$ 0.020	& 3	\\
% H\dotfill			& 2MASS H mag.\dotfill & 11.324 $\pm$ 0.026	    &  3	\\
% K$_{\rm S}$\dotfill			& 2MASS ${\rm K_S}$ mag.\dotfill & 11.226 $\pm$ 0.021 &  3	\\
% %\\
% W1\dotfill		& WISE1 mag.\dotfill & 11.135 $\pm$ 0.023 & 4	\\
% W2\dotfill		& WISE2 mag.\dotfill & 11.155 $\pm$ 0.020 &  4 \\
% W3\dotfill		& WISE3 mag.\dotfill & 11.160 $\pm$ 0.086& 4	\\
% W4\dotfill		& WISE4 mag.\dotfill & 9.246 $\pm$ N/A &  4	\\
\\
$\pi$\dotfill & Gaia EDR3 parallax (mas) \dotfill & 2.351 $\pm$ 0.089 &  1 \\
$d$\dotfill & Distance (pc)\dotfill & $425.3 \pm 16.1$ & 1 \\
$\mu_{\alpha'}$\dotfill		& Gaia EDR3 proper motion\dotfill & -5.387 $\pm$ 0.104 & 1 \\
                    & \hspace{3pt} in RA (mas yr$^{-1}$)	&  \\
$\mu_{\delta}$\dotfill		& Gaia EDR3 proper motion\dotfill 	&  11.349 $\pm$ 0.096 &  1 \\
                    & \hspace{3pt} in DEC (mas yr$^{-1}$) &  \\
RUWE\dotfill		& Gaia EDR3 renormalized\dotfill 	&  1.120 &  1 \\
                    & \hspace{3pt} unit weight error &  \\
% RV\dotfill & Gaia EDR3 systemic \hspace{9pt}\dotfill  & $17.44 \pm 0.64$$^{\dagger}$ & 1 \\
%                     & \hspace{3pt} radial velocity (\kms)  & \\
% RV\dotfill & Adopted systemic \hspace{9pt}\dotfill  & $17.44 \pm 0.64$$^{\dagger}$ & 1 \\
%                     & \hspace{3pt} radial velocity (\kms)  & \\
%
% \\
% $v\sin{i_\star}$\dotfill &  Rotational velocity (\kms) \hspace{9pt}\dotfill &  -- $\pm$ -- & 5 \\
% $v_{\rm mac}$\dotfill &  Macroturbulence velocity (\kms) \hspace{9pt}\dotfill &  -- $\pm$ -- & 5 \\
${\rm [Fe/H]}$\dotfill &   Metallicity \hspace{9pt}\dotfill & -- $\pm$ -- & 5 \\
$T_{\rm eff}$\dotfill &  Effective Temperature (K) \hspace{9pt}\dotfill & ---- $\pm$ --- &  6  \\
$\log{g_{\star}}$\dotfill &  Surface Gravity (cgs)\hspace{9pt}\dotfill &  x.xxx $\pm$ x.xx  &  6 \\
%
Li EW\dotfill & 6708\AA\ Equiv{.} Width (m\AA) \dotfill & NaN  & 7 \\
%
$P_{\rm rot}$\dotfill & Rotation period (d)\dotfill & NaN  & 8 \\
Age & Adopted stellar age (Myr)\dotfill & ---  &  9 \\
% $E(B-V)$\dotfill & Reddening (mag)\dotfill & $0.06 \pm 0.02$ & 9 \\
%
Spec. Type\dotfill & Spectral Type\dotfill & 	M1V{\bf FIX} & 5 \\
%
$R_\star$\dotfill & Stellar radius ($R_\odot$)\dotfill & 0.XXX$\pm$X.XXX & 6 \\
$M_\star$\dotfill & Stellar mass ($R_\odot$)\dotfill & 0.XXX$\pm$X.XXX & 6 \\
%$F_{\rm bol}$\dotfill & Stellar bolometric flux (cgs)\dotfill & (1.967$\pm$0.046)$\times10^{-9}$ & 9 \\
$A_{\rm V}$\dotfill & Interstellar reddening (mag)\dotfill & 0.XX$\pm$0.XX & 10 \\
% $U^{*}$\dotfill & Space Velocity (\kms)\dotfill & $26.24 \pm 0.46$  & \S\ref{sec:uvw} \\
% $V$\dotfill       & Space Velocity (\kms)\dotfill & $-71.52 \pm 1.68$ & \S\ref{sec:uvw} \\
% $W$\dotfill       & Space Velocity (\kms)\dotfill & $ -1.31 \pm 0.27$ & \S\ref{sec:uvw} \\
\hline
\end{tabular}
\begin{flushleft}
 \footnotesize{ \textsc{NOTE}---
% $\dagger$ Systemic RV uncertainty is the standard deviation of single-transit radial velocities, as quoted in Gaia DR2. %$*$ $U$ is in the direction of the Galactic center. \\
  {\bf FIXME}
Provenances are:
$^1$\citet{gaia_collaboration_gaia_2018},
$^2$\citet{stassun_TIC8_2019},
$^3$\citet{skrutskie_tmass_2006},
$^4$\citet{wright_WISE_2010},
$^5$CHIRON spectra,
$^6$Method~2 (cluster isochrone, Section~\ref{subsec:starparams}),
$^7$FEROS spectra,
$^8$TESS light curve,
$^9$IC~2602 ages from isochrone \& lithium depletion analyses (Section~\ref{subsec:clusterchar}),
$^{10}$Method~1 (photometric SED fit, Section~\ref{subsec:starparams}).}
\end{flushleft}
\vspace{-0.5cm}
\end{table*}

% \input{model_posterior_table.tex}

\clearpage
\bibliographystyle{yahapj}                            
\bibliography{bibliography} 

\appendix
\section{Clustering method details}
\label{app:clustering}

%
% TODO: the following two paragraphs, when quoted, are ENTIRELY FROM
% the relevant papers. they should probably be nixed in favor of
% concision.
%
\citetalias{cantatgaudin_gaia_2018} applied a procedure that ultimately
yielded what we call ``the core''.  Their procedure was to first query
a Gaia DR2 cone around the previously reported RA and dec of the
cluster, and within $\pm 0.5$\,mas of its previously reported
distance. 
% TODO: which was it?
The outer radius of their cone was either r2 from MWSC (CITE
Kharchenko 2013), or twice the ``cluster radius'' listed by Dias.  No
proper motion cut was applied.  They then applied an unsupervised
classification scheme called UPMASK to $G<18$\,mag stars within this
cone (CITE).  The steps of UPMASK are first to perform a k-means
clustering in the ``astrometric space'' ($\mu_{\alpha'}, \mu_\delta,
\pi$).  Then, a ``veto'' step is applied to assess whether the groups
of stars output from the k-means clustering are or are not more
concentrated than a random distribution.  This is implemented by
``comparing the total branch length of the minimum spanning tree
connecting the stars with the expected branch length for a random
uniform distribution covering the investigated field of view''.  ``To
turn this yes/no flag to a membership probability,
\citet{cantatgaudin_gaia_2018} then redraw new values of
($\mu_{\alpha'}, \mu_\delta, \pi$) for each source based on the listed
value, uncertainty, and covariance.  After a certain number of
redrawings, the final probability is the frequency with which a given
star passes the veto".  In the case of \cn, this yielded a reported
``\texttt{r50}'' within which half of the cluster members were found
to be within 0.496$^\circ$.
When we selected candidate \cn members from the results of
\citetalias{cantatgaudin_gaia_2018}, we opted to include all candidate members
with reported membership probability exceeding 10\%.
While this seems {\it a priori} low, our results ({\bf SECTION XXX})
show that this ``membership probabilty'' severely underestimates the
purity of the \citetalias{cantatgaudin_gaia_2018} sample for \cn. 
%TODO: quantify
Their false positive rate across the sample is more like 1-5\%.

\citetalias{kounkel_untangling_2019}
applied a different unsupervised
clustering method to the 5-dimensional Gaia DR2 data (omitting radial
velocities, due to their sparsity).  Their selection function (see
their Section~2) yielded $\approx 2\times 10^7$ stars, mostly within
$\approx 1$\,kpc and typically with $G<18$\,mag.  Their clustering
algorithm was the ``hierarchical density-based spatial clustering of
applications with noise'' (HDBSCAN, CITE McInnes17).  The classical
DBSCAN algorithm ``identifies clusters as overdensities in a
multi-dimensional space in which the number of sources exceeds the
required minimum number of points within a neighborhood of a
particular linking length $\epsilon$.  HDBSCAN does not depend on
$\epsilon$; instead it condenses the minimum spanning tree by pruning
off the nodes that do not meet the minimum number of sources in a
cluster and reanalyzing the nodes that do. Depending on the chosen
algorithm, it would then either find the most persistent structure
(through the excess of mass method), or return clusters as the leaves
of the tree (which results in somewhat more homogeneous clusters). In
both cases it is more effective at finding structures of varying
densities in a given data set than DBSCAN.''
`` The two main parameters that control HDBSCAN are the number of
sources in a cluster and the number of samples. The former is the
parameter that rejects groupings that are too small; the latter sets
the threshold of how conservative the algorithm is in its
considerations of the background noise (even if the resulting noisy
groupings do meet the minimum cluster size).  By default, the sample
size is set to the same value as the cluster size, but it is possible
to adjust them separately.''
Regarding membership proabilities,
\citetalias{kounkel_untangling_2019} did not report continuous membership
probabilities, instead opting for the binary ``member'' or ``not''.



\section{Detrending details}
\label{app:detrending}

In ``detrending'' for our general variability search, our goal was to
preserve astrophysical variability, while removing systematic
variability.  One particular concern for the TESS light curves is
systematic variability at the timescale of the 14-day satellite orbit,
mostly induced by scattered light from the Earth and Moon.

We therefore turned to the principal components (i.e., the
eigenvectors) calculated following the procedure described by
\citet{bouma_cdipsI_2019}. In brief, these vectors are computed using
a set of ``trend stars'' selected from across each CCD according to
ad-hoc heuristics that (hopefully) lead them to be dominated by {\it
systematic} variability (Sec~3.7.2).

The principal component vectors, also referred to as the eigenvectors,
are rank-ordered by the degree of variance that they predict in the
training set (of ``trend stars'').

We then posit that any given target star's light curve is described as
a linear combination of the eigenvectors.  Optionally, we also
considered the inclusion of additional systematic vectors that could
affect the light curve, such as the CCD temperature, the flux level
measured in the background annulus, and the centroid positions of the
stars on the CCDs.  These can be treated as additional ``features'' in
the linear model.

There are many methods for determining the coefficients of the linear
model, after the full set of eigenvectors (plus optionally
``sytematic'' vectors) has been asssembled.  We explored two: ordinary
least squares, and ridge regression. Ridge regression is the same as
ordinary least squares, except it includes an L2 norm with a regularization
coefficient. The regularization coefficient that best applied for any
given target light curve was solved for using a cross-validation grid
search, using \texttt{sklearn.linear\_modelRidgeCV} (CITE). 

Each target light curve was mean-subtracted and normalized by its
standard deviation, as were the eigenvectors. The linear problem was
then solved, and the light curve was reconstructed by re-adding the
original mean, and re-multiplying by the standard deviation to ensure
that the variance of the light curve did not change.

We found, somewhat to our surprise, that the choice of using ordinary
least squares versus ridge regression did not seem to significantly
affect the resulting light curves. In other words, the inclusion (or
lack thereof) of a regularization did not strongly alter the
best-fitting coefficients.

A few other choices seemed to be more important:

\begin{itemize}

  \item {\it To smooth, or to not smooth the eigenvectors}.
  Ideally, the eigenvectors should be smooth (and not contain
    residuals from {\it e.g.}, eclipsing binaries that snuck their way
    into the template set). However this does not always apply.
    Sometimes, the eigenvectors are also noisy, which leads them to
    induce extra variability into the PCA ``detrended'' light curves.
    To address this problem, we opted to smooth the eigenvectors using
    a windowed filter (with a "biweight" weight scheme, implemented in
    \texttt{wotan} by \citet{hippke_wotan_2019}; window length 1 day, cval 6).
  One issue with this is that systematic sharp features (captured e.g., in
  "spike vectors") no longer are captured, so they end up in the "PCA
  detrended" light curves. They can be filtered out relatively easily though
  (using rolling outlier rejection), and we prefer this approach to having
  systematic features {\it injected by} the PCA detrending.

  \item {\it How many eigenvectors to use}.
    A larger number always leads to greater whitening.  In
    \citet{bouma_cdipsI_2019}, we performed a Factor Analysis
    cross-validation to determine the number of eigenvectors to use.
    The typical number adopted based on this analysis was 10--15.
    Even though this approach was chosen to prevent over-fitting, in
    our experience, for stellar rotation it still often lead to
    overfitting, especially for rotation signals with periodicities of
    $\gtrsim 3$ days.  (Shorter signals typically are not distorted,
    since the eigenvectors generally do not contain the high-frequency
    content that leads to the distortions).  For this analysis, we
    therefore impose the maximum number of eigenvectors to be 5.

  \item {\it Which supplementary systematics vectors to use}.
    We considered using the \texttt{BGV}, \texttt{CCDTEMP},
    \texttt{XIC}, \texttt{YIC}, and \texttt{BGV} vectors, packaged
    with the CDIPS light curves. we found that the background value
    measured in an annulus centered on the aperture, \texttt{BGV},
    tended to produce the best independent information from the PCA
    eigenvectors, and so we adopted it as our only ``supplementary''
    trend vector.  We opted to not smooth it (in hopes that it would
    provide direct complement to the smoothed PCA vectors; 1 sharp
    vector containing literally the background information, plus 5
    smooth vectors).
\end{itemize}

For every ``target star'', we then decorrelated the raw
(image-subtracted and background-subtracted) light curve
using a linear model with ordinary least squares.


\section{Rotation $\otimes$ RUWE}

\begin{figure*}[t]
	\begin{center}
		\leavevmode
		\includegraphics[width=0.95\textwidth]{f6.pdf}
	\end{center}
	\vspace{-0.7cm}
	\caption{ {\bf Rotation versus color, colored by RUWE.}
		Looks like on the slow sequence, there's more yellow at the bottom?
		\label{fig:rotn_X_RUWE}
	}
\end{figure*}




\listofchanges

%\allauthors
\end{document}
