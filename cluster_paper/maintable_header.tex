%% \begin{deluxetable}{} command tell LaTeX how many columns
%% there are and how to align them.
%\startlongtable
\begin{deluxetable*}{lll}
    
%% Keep a portrait orientation

%% Over-ride the default font size
%% Use Default (12pt)
%\tabletypesize{\scriptsize}
\tabletypesize{\small}

%% Use \tablewidth{?pt} to over-ride the default table width.
%% If you are unhappy with the default look at the end of the
%% *.log file to see what the default was set at before adjusting
%% this value.

%% This is the title of the table.
\tablecaption{Rotation periods and lithium equivalent widths for
\nkinematic\ candidate NGC\,2516 members.}
\label{tab:maintable}

%% This command over-rides LaTeX's natural table count
%% and replaces it with this number.  LaTeX will increment 
%% all other tables after this table based on this number
%\tablenum{3}

%% The \tablehead gives provides the column headers.  It
%% is currently set up so that the column labels are on the
%% top line and the units surrounded by ()s are in the 
%% bottom line.  You may add more header information by writing
%% another line between these lines. For each column that requries
%% extra information be sure to include a \colhead{text} command
%% and remember to end any extra lines with \\ and include the 
%% correct number of &s.
\tablehead{
  \colhead{Parameter} &
  \colhead{Example Value} &
  \colhead{Description}
}

%% All data must appear between the \startdata and \enddata commands
%
% paste from
% /Users/luke/Dropbox/proj/earhart/results/tables/NGC_2516_Prot_cleaned_header.tex
% via drivers/write_NGC2516_main_table.py
\startdata
               \texttt{source\_id} & 5290665204845694336 &                                                                   Gaia DR2 source identifier. \\
         \texttt{source\_id\_edr3} & 5290665204845694336 &                                                                  Gaia EDR3 source identifier. \\
                 \texttt{in\_SetA} &                True &                                 In Set $\mathcal{A}$ (LSP>0.08, P<15d, nequal==0, nclose>=1). \\
                 \texttt{in\_SetB} &                True &                                   In Set $\mathcal{B}$ (Set $\mathcal{A}$ and periods match). \\
         \texttt{n\_cdips\_sector} &                   7 &                                               Number of TESS sectors with CDIPS light curves. \\
                   \texttt{period} &            0.228981 &                                                              Lomb-Scargle best period [days]. \\
                   \texttt{lspval} &            0.657826 &                                               Lomb-Scargle periodogram value for best period. \\
               \texttt{spdmperiod} &            0.228981 &                                                          Stellingwerf PDM best period [days]. \\
                  \texttt{spdmval} &            0.372779 &                                           Stellingwerf PDM periodogram value for best period. \\
                   \texttt{nequal} &                   0 &                                    Number of stars brighter than the target in TESS aperture. \\
                   \texttt{nclose} &                   1 &                                      Number of stars with $\Delta T > 1.25$ in TESS aperture. \\
                   \texttt{nfaint} &                   1 &                                       Number of stars with $\Delta T > 2.5$ in TESS aperture. \\
                       \texttt{ra} &          119.234778 &                                                               Gaia DR2 right ascension [deg]. \\
                      \texttt{dec} &          -61.015929 &                                                                   Gaia DR2 declination [deg]. \\
               \texttt{ref\_epoch} &              2015.5 &                                          Reference epoch for right ascension and declination. \\
                 \texttt{parallax} &            2.458023 &                                                                      Gaia DR2 parallax [mas]. \\
          \texttt{parallax\_error} &            0.028395 &                                                          Gaia DR2 parallax uncertainty [mas]. \\
                     \texttt{pmra} &           -4.225826 &                           Gaia DR2 proper motion $\mu_\alpha \cos \delta$ [mas$\,$yr$^{-1}$]. \\
                    \texttt{pmdec} &           10.899938 &                                       Gaia DR2 proper motion $\mu_\delta$ [mas$\,$yr$^{-1}$]. \\
       \texttt{phot\_g\_mean\_mag} &           15.778953 &                                                                       Gaia DR2 $G$ magnitude. \\
      \texttt{phot\_bp\_mean\_mag} &           16.629477 &                                                           Gaia DR2 $G_\mathrm{BP}$ magnitude. \\
      \texttt{phot\_rp\_mean\_mag} &            14.85896 &                                                           Gaia DR2 $G_\mathrm{RP}$ magnitude. \\
         \texttt{radial\_velocity} &                 NaN &                                       Gaia DR2 heliocentric radial velocity [km$\,$s$^{-1}$]. \\
  \texttt{radial\_velocity\_error} &                 NaN &                                        Gaia DR2 radial velocity uncertainty [km$\,$s$^{-1}$]. \\
               \texttt{subcluster} &                core &                                                    Is star in core (CG18) or halo (KC19+M21)? \\
                 \texttt{in\_CG18} &                True &                                                       Star in \citet{cantatgaudin_gaia_2018}. \\
                 \texttt{in\_KC19} &                True &                                                      Star in \citet{kounkel_untangling_2019}. \\
                  \texttt{in\_M21} &               False &                                                                Star in \citet{meingast_2021}. \\
               \texttt{(Bp-Rp)\_0} &            1.636198 & Gaia $G_\mathrm{BP}$-$G_\mathrm{RP}$ color, minus $E$($G_\mathrm{BP}$-$G_\mathrm{RP}$)=0.1343 \\
            \texttt{is\_phot\_bin} &               False &                                                   True if $>0.3$ mag above cluster isochrone. \\
           \texttt{is\_astrm\_bin} &               False &                                                                 True if Gaia EDR3 RUWE > 1.2. \\
                     \texttt{ruwe} &            0.986349 &                                                                               Gaia EDR3 RUWE. \\
      \texttt{Li\_EW\_mA\_GaiaESO} &              -1.495 &                          Gaia-ESO Li doublet equivalent width, including the Fe blend [m\AA]. \\
\texttt{Li\_EW\_mA\_perr\_GaiaESO} &              12.407 &                                              Gaia-ESO Li doublet EW upper uncertainty [m\AA]. \\
\texttt{Li\_EW\_mA\_merr\_GaiaESO} &                 5.0 &                                              Gaia-ESO Li doublet EW lower uncertainty [m\AA]. \\
        \texttt{Li\_EW\_mA\_GALAH} &                 NaN &                             GALAH Li doublet equivalent width, including the Fe blend [m\AA]. \\
  \texttt{Li\_EW\_mA\_perr\_GALAH} &                 NaN &                                                 GALAH Li doublet EW upper uncertainty [m\AA]. \\
  \texttt{Li\_EW\_mA\_merr\_GALAH} &                 NaN &                                                 GALAH Li doublet EW lower uncertainty [m\AA]. \\
\enddata

%% Include any \tablenotetext{key}{text}, \tablerefs{ref list},
%% or \tablecomments{text} between the \enddata and 
%% \end{deluxetable} commands

%% General table comment marker
\tablecomments{
Table~\ref{tab:maintable} is published in its entirety in a
machine-readable format.  One entry is shown for guidance regarding
form and content.  This table is a concatenation of all candidate
NGC\,2516 members reported by \citetalias{cantatgaudin_gaia_2018},
\citetalias{kounkel_untangling_2019}, and \citetalias{meingast_2021}
based on the Gaia DR2 data.  Different levels of purity and
completeness can be achieved using different cuts on photometric
periods, periodogram powers, and lithium eqiuvalent widths.  Sets
$\mathcal{A}$ and $\mathcal{B}$ provide two possible levels of
cleaning (see Section~\ref{subsubsec:cluster}).  When the
target star is the only star present in the TESS aperture,
\texttt{nequal}$=0$, \texttt{nclose}$=1$, and \texttt{nfaint}$=1$.
The light curves are available at
\url{https://archive.stsci.edu/hlsp/cdips}.
Supplementary plots enabling the analysis individual stars
are available at \url{https://lgbouma.com/notes}.
}
\vspace{-0.5cm}
\end{deluxetable*}
