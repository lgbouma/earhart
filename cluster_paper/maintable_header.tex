%% \begin{deluxetable}{} command tell LaTeX how many columns
%% there are and how to align them.
%\startlongtable
\begin{deluxetable*}{lll}
    
%% Keep a portrait orientation

%% Over-ride the default font size
%% Use Default (12pt)
\tabletypesize{\scriptsize}
%\tabletypesize{\small}

%% Use \tablewidth{?pt} to over-ride the default table width.
%% If you are unhappy with the default look at the end of the
%% *.log file to see what the default was set at before adjusting
%% this value.

%% This is the title of the table.
\tablecaption{Rotation periods and lithium equivalent widths for
candidate NGC\,2516 members.}
\label{tab:maintable}

%% This command over-rides LaTeX's natural table count
%% and replaces it with this number.  LaTeX will increment 
%% all other tables after this table based on this number
%\tablenum{3}

%% The \tablehead gives provides the column headers.  It
%% is currently set up so that the column labels are on the
%% top line and the units surrounded by ()s are in the 
%% bottom line.  You may add more header information by writing
%% another line between these lines. For each column that requries
%% extra information be sure to include a \colhead{text} command
%% and remember to end any extra lines with \\ and include the 
%% correct number of &s.
\tablehead{
  \colhead{Parameter} &
  \colhead{Example Value} &
  \colhead{Description}
}

%% All data must appear between the \startdata and \enddata commands
%
% paste from
% /Users/luke/Dropbox/proj/earhart/results/tables/NGC_2516_Prot_cleaned_header.tex
% via drivers/write_NGC2516_main_table.py
\startdata
             \texttt{source\_id} & 5290824530952816896 &                                         Gaia DR2 source identifier. \\
       \texttt{n\_cdips\_sector} &                   3 &                     Number of TESS sectors with CDIPS light curves. \\
                 \texttt{period} &            1.297662 &                                           Lomb-Scargle best period. \\
                 \texttt{lspval} &            0.171629 &                     Lomb-Scargle periodogram value for best period. \\
                 \texttt{nequal} &                 0.0 &          Number of stars brighter than the target in TESS aperture. \\
                 \texttt{nclose} &                 1.0 &            Number of stars with $\Delta T > 1.25$ in TESS aperture. \\
                 \texttt{nfaint} &                 1.0 &             Number of stars with $\Delta T > 2.5$ in TESS aperture. \\
             \texttt{spdmperiod} &            1.297662 &                                       Stellingwerf PDM best period. \\
                \texttt{spdmval} &            0.826804 &                 Stellingwerf PDM periodogram value for best period. \\
             \texttt{ref\_epoch} &              2015.5 &                                         Reference epoch for RA/dec. \\
                     \texttt{ra} &          120.125821 &                                           Gaia DR2 right ascension. \\
                    \texttt{dec} &          -60.522668 &                                               Gaia DR2 declination. \\
               \texttt{parallax} &            1.980389 &                                                  Gaia DR2 parallax. \\
        \texttt{parallax\_error} &            0.031363 &                                      Gaia DR2 parallax uncertainty. \\
                   \texttt{pmra} &           -4.480769 &                    Gaia DR2 proper motion $\mu_\alpha \cos \delta$. \\
                  \texttt{pmdec} &            8.203306 &                                Gaia DR2 proper motion $\mu_\delta$. \\
     \texttt{phot\_g\_mean\_mag} &           12.958735 &                                             Gaia DR2 $G$ magnitude. \\
    \texttt{phot\_bp\_mean\_mag} &           13.309761 &                                 Gaia DR2 $G_\mathrm{BP}$ magnitude. \\
    \texttt{phot\_rp\_mean\_mag} &           12.406217 &                                 Gaia DR2 $G_\mathrm{RP}$ magnitude. \\
       \texttt{radial\_velocity} &           -0.227623 &                              Gaia DR2 heliocentric radial velocity. \\
\texttt{radial\_velocity\_error} &            1.945767 &                               Gaia DR2 radial velocity uncertainty. \\
             \texttt{subcluster} &                halo &                          Is star in core (CG18) or halo (KC19+M21)? \\
               \texttt{in\_CG18} &               False &                             Star in \citet{cantatgaudin_gaia_2018}. \\
               \texttt{in\_KC19} &                True &                            Star in \citet{kounkel_untangling_2019}. \\
                \texttt{in\_M21} &               False &                                      Star in \citet{meingast_2021}. \\
             \texttt{(Bp-Rp)\_0} &            0.769225 & Gaia Bp-Rp color, minus $E$($G_\mathrm{BP}$-$G_\mathrm{RP}$)=0.1343 \\
               \texttt{in\_SetA} &                True &       In Set $\mathcal{A}$ (LSP>0.08, P<15d, nequal==0, nclose>=1). \\
               \texttt{in\_SetB} &                True &         In Set $\mathcal{B}$ (Set $\mathcal{A}$ and periods match). \\
          \texttt{is\_phot\_bin} &                True &                         True if $>0.3$ mag above cluster isochrone. \\
         \texttt{is\_astrm\_bin} &                True &                                                 True if RUWE > 1.2. \\
\enddata

%% Include any \tablenotetext{key}{text}, \tablerefs{ref list},
%% or \tablecomments{text} between the \enddata and 
%% \end{deluxetable} commands

%% General table comment marker
\tablecomments{
Table~\ref{tab:maintable} is published in its entirety in a
machine-readable format.  One entry is shown for guidance regarding
form and content.  This table is a concatenation of all candidate
NGC\,2516 members reported by \citetalias{cantatgaudin_gaia_2018},
\citetalias{kounkel_untangling_2019}, and \citetalias{meingast_2021}
based on the Gaia DR2 data.  Different levels of purity and
completeness can be achieved using different cuts on photometric
periods and lithium eqiuvalent widths (see Figure~\ref{fig:rot}).
Sets $\mathcal{A}$ and $\mathcal{B}$ provide two possible levels of
cleaning based on rotation periods.  Note that when the target star is
the only star present in the TESS aperture, \texttt{nequal}$=0$,
\texttt{nclose}$=1$, and \texttt{nfaint}$=1$.  The light curves are
available at \url{https://archive.stsci.edu/hlsp/cdips}.
}
\vspace{-0.5cm}
\end{deluxetable*}
