\begin{table*}
\scriptsize
\setlength{\tabcolsep}{2pt}
\centering
\caption{Literature and Measured Properties for TOI$\,$1937B}
\label{tab:compparams}
%\tablenum{2}
\begin{tabular}{llcc}
  \hline
  \hline
Other identifiers\dotfill & \\
\multicolumn{3}{c}{TIC 766593811} \\
\multicolumn{3}{c}{GAIADR2 5489726768531118848} \\
\multicolumn{3}{c}{GAIAEDR3 5489726768531118848} \\
\hline
\hline
Parameter & Description & Value & Source\\
\hline 
$\alpha_{J2016.0}$\dotfill	&Right Ascension (deg)\dotfill & 116.3706 $\pm$ 0.0098& 1	\\
$\delta_{J2016.0}$\dotfill	&Declination (deg)\dotfill & -52.3826 $\pm$ 0.0753  & 1	\\
% $l_{J2015.5}$\dotfill	&Galactic Longitude (deg)\dotfill & 265.3082 & 1	\\
% $b_{J2015.5}$\dotfill	&Galactic Latitude (deg)\dotfill & -13.5487 & 1	\\
%\\
%$NUV$\dotfill           & GALEX $NUV$ mag.\dotfill & 13.804 $\pm$ 0.004 & 2 \\
%$FUV$\dotfill           & GALEX $FUV$ mag.\dotfill & 18.466 $\pm$ 0.056 & 2 \\
\\
%B\dotfill			&Johnson B mag.\dotfill & 11.119 $\pm$ 0.107		& 2	\\
%V\dotfill			&Johnson V mag.\dotfill & 13.18 $\pm$ 0.10		& 2	\\
%$B$\tablenote{The uncertainties of the photometry have a systematic error floor applied. Even still, the global fit requires a significant scaling of the uncertainties quoted here to be consistent with our model, suggesting they are still significantly underestimated for one or more of the broad band magnitudes}\dotfill		& APASS Johnson $B$ mag.\dotfill	& 13.001 $\pm$	0.02& 2	\\
%$V$\dotfill		& APASS Johnson $V$ mag.\dotfill	& 11.808 $\pm$	0.02& 2	\\
%\\
${\rm G}$\dotfill     & Gaia $G$ mag.\dotfill     & 17.653$\pm$0.003 & 1\\
${\rm Bp}$\dotfill     & Gaia $Bp$ mag.\dotfill     & 17.950 $\pm$0.098 & 1\\
${\rm Rp}$\dotfill     & Gaia $Rp$ mag.\dotfill     & 16.246 $\pm$0.015 & 1\\
${\rm T}$\dotfill     & TESS mag.\dotfill     & 16.86$\pm$0.08 & 2\\
$\Delta I_{\rm C}$\dotfill     & SOAR Cousins-I mag diff.\dotfill & 4.3$\pm$0.X & 2\\
%$u'$\dotfill        & Sloan $u'$ mag.\dotfill & 14.706 $\pm$ 0.006& 3\\
%$g'$\dotfill		& APASS Sloan $g'$ mag.\dotfill	& 12.407 $\pm$ 0.02	&  2	\\
%$r'$\dotfill		& APASS Sloan $r'$ mag.\dotfill	& 11.311 $\pm$ 0.02	&  2	\\
%$i'$\dotfill		& APASS Sloan $i'$ mag.\dotfill	& 10.927 $\pm$ 0.04 &  2	\\
%\\
% J\dotfill			& 2MASS J mag.\dotfill & 11.717  $\pm$ 0.020	& 3	\\
% H\dotfill			& 2MASS H mag.\dotfill & 11.324 $\pm$ 0.026	    &  3	\\
% K$_{\rm S}$\dotfill			& 2MASS ${\rm K_S}$ mag.\dotfill & 11.226 $\pm$ 0.021 &  3	\\
% %\\
% W1\dotfill		& WISE1 mag.\dotfill & 11.135 $\pm$ 0.023 & 4	\\
% W2\dotfill		& WISE2 mag.\dotfill & 11.155 $\pm$ 0.020 &  4 \\
% W3\dotfill		& WISE3 mag.\dotfill & 11.160 $\pm$ 0.086& 4	\\
% W4\dotfill		& WISE4 mag.\dotfill & 9.246 $\pm$ N/A &  4	\\
\\
$\pi$\dotfill & Gaia EDR3 parallax (mas) \dotfill & 2.351 $\pm$ 0.089 &  1 \\
$d$\dotfill & Distance (pc)\dotfill & $425.3 \pm 16.1$ & 1 \\
$\mu_{\alpha'}$\dotfill		& Gaia EDR3 proper motion\dotfill & -5.387 $\pm$ 0.104 & 1 \\
                    & \hspace{3pt} in RA (mas yr$^{-1}$)	&  \\
$\mu_{\delta}$\dotfill		& Gaia EDR3 proper motion\dotfill 	&  11.349 $\pm$ 0.096 &  1 \\
                    & \hspace{3pt} in DEC (mas yr$^{-1}$) &  \\
RUWE\dotfill		& Gaia EDR3 renormalized\dotfill 	&  1.120 &  1 \\
                    & \hspace{3pt} unit weight error &  \\
% RV\dotfill & Gaia EDR3 systemic \hspace{9pt}\dotfill  & $17.44 \pm 0.64$$^{\dagger}$ & 1 \\
%                     & \hspace{3pt} radial velocity (\kms)  & \\
% RV\dotfill & Adopted systemic \hspace{9pt}\dotfill  & $17.44 \pm 0.64$$^{\dagger}$ & 1 \\
%                     & \hspace{3pt} radial velocity (\kms)  & \\
%
% \\
% $v\sin{i_\star}$\dotfill &  Rotational velocity (\kms) \hspace{9pt}\dotfill &  -- $\pm$ -- & 5 \\
% $v_{\rm mac}$\dotfill &  Macroturbulence velocity (\kms) \hspace{9pt}\dotfill &  -- $\pm$ -- & 5 \\
${\rm [Fe/H]}$\dotfill &   Metallicity \hspace{9pt}\dotfill & -- $\pm$ -- & 5 \\
$T_{\rm eff}$\dotfill &  Effective Temperature (K) \hspace{9pt}\dotfill & ---- $\pm$ --- &  6  \\
$\log{g_{\star}}$\dotfill &  Surface Gravity (cgs)\hspace{9pt}\dotfill &  x.xxx $\pm$ x.xx  &  6 \\
%
Li EW\dotfill & 6708\AA\ Equiv{.} Width (m\AA) \dotfill & NaN  & 7 \\
%
$P_{\rm rot}$\dotfill & Rotation period (d)\dotfill & NaN  & 8 \\
Age & Adopted stellar age (Myr)\dotfill & ---  &  9 \\
% $E(B-V)$\dotfill & Reddening (mag)\dotfill & $0.06 \pm 0.02$ & 9 \\
%
Spec. Type\dotfill & Spectral Type\dotfill & 	M1V{\bf FIX} & 5 \\
%
$R_\star$\dotfill & Stellar radius ($R_\odot$)\dotfill & 0.XXX$\pm$X.XXX & 6 \\
$M_\star$\dotfill & Stellar mass ($R_\odot$)\dotfill & 0.XXX$\pm$X.XXX & 6 \\
%$F_{\rm bol}$\dotfill & Stellar bolometric flux (cgs)\dotfill & (1.967$\pm$0.046)$\times10^{-9}$ & 9 \\
$A_{\rm V}$\dotfill & Interstellar reddening (mag)\dotfill & 0.XX$\pm$0.XX & 10 \\
% $U^{*}$\dotfill & Space Velocity (\kms)\dotfill & $26.24 \pm 0.46$  & \S\ref{sec:uvw} \\
% $V$\dotfill       & Space Velocity (\kms)\dotfill & $-71.52 \pm 1.68$ & \S\ref{sec:uvw} \\
% $W$\dotfill       & Space Velocity (\kms)\dotfill & $ -1.31 \pm 0.27$ & \S\ref{sec:uvw} \\
\hline
\end{tabular}
\begin{flushleft}
 \footnotesize{ \textsc{NOTE}---
% $\dagger$ Systemic RV uncertainty is the standard deviation of single-transit radial velocities, as quoted in Gaia DR2. %$*$ $U$ is in the direction of the Galactic center. \\
  {\bf FIXME}
Provenances are:
$^1$\citet{gaia_collaboration_gaia_2018},
$^2$\citet{stassun_TIC8_2019},
$^3$\citet{skrutskie_tmass_2006},
$^4$\citet{wright_WISE_2010},
$^5$CHIRON spectra,
$^6$Method~2 (cluster isochrone, Section~\ref{subsec:starparams}),
$^7$FEROS spectra,
$^8$TESS light curve,
$^9$IC~2602 ages from isochrone \& lithium depletion analyses (Section~\ref{subsec:clusterchar}),
$^{10}$Method~1 (photometric SED fit, Section~\ref{subsec:starparams}).}
\end{flushleft}
\vspace{-0.5cm}
\end{table*}
