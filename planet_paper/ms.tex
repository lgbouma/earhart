%%%%%%%%%%%%%%%%%%%%%%%%%%%%%%%%%%%%%%%%%%%%%%%%%%%%%%%%%%%%%%%%%%%%%%%%%%%%%%%

\documentclass[12pt,twocolumn,tighten]{aastex63}
%\documentclass[12pt,twocolumn,tighten,trackchanges]{aastex63}
\usepackage{amsmath,amstext,amssymb}
\usepackage[T1]{fontenc}
\usepackage{apjfonts}
\usepackage[figure,figure*]{hypcap}
\usepackage{graphics,graphicx}
\usepackage{hyperref}
\usepackage{natbib}
\usepackage[caption=false]{subfig} % for subfloat
\usepackage{enumitem} % for specific spacing of enumerate
\usepackage{epigraph}

\renewcommand*{\sectionautorefname}{Section} %for \autoref
\renewcommand*{\subsectionautorefname}{Section} %for \autoref

\newcommand{\tn}{TOI~1937} % target star name
\newcommand{\pn}{TOI~1937b} % planet name
\newcommand{\cn}{NGC~2516} % cluster name

\newcommand{\kms}{\,km\,s$^{-1}$}

\newcommand{\stscilink}{\textsc{\url{archive.stsci.edu/hlsp/cdips}}}
\newcommand{\datasetlink}{\textsc{\dataset[doi.org/10.17909/t9-ayd0-k727]{https://doi.org/10.17909/t9-ayd0-k727}}}

%% Reintroduced the \received and \accepted commands from AASTeX v5.2.
%% Add "Submitted to " argument.
\received{---}
\revised{---}
\accepted{---}
\submitjournal{TBD.}%AAS journals.}
\shorttitle{TDB.}

\begin{document}

\defcitealias{bouma_wasp4b_2019}{B19}

\title{
  Cluster Difference Imaging Photometric Survey. IV.
  TOI 1937Ab. Youngest Hot Jupiter, or Just Tidally Spun Up?
}

%\suppressAffiliations
%\NewPageAfterKeywords
\correspondingauthor{L.\,G.\,Bouma}
\email{luke@astro.princeton.edu}

%
% key contributions
%
\author[0000-0002-0514-5538]{L. G. Bouma}
\affiliation{Department of Astrophysical Sciences, Princeton University, 4 Ivy Lane, Princeton, NJ 08540, USA}

\author[0000-0001-8732-6166]{J. D. Hartman}
\affiliation{Department of Astrophysical Sciences, Princeton University, 4 Ivy Lane, Princeton, NJ 08540, USA}

% PENDING
\author[0000-0001-7409-5688]{G. Stefansson} % halpha, bisectors, spec analysis
\affiliation{Department of Astrophysical Sciences, Princeton University, 4 Ivy Lane, Princeton, NJ 08540, USA}

%PENDING
\author[0000-0002-4891-3517]{G. Zhou} % metallicity analysis++
\affiliation{Center for Astrophysics \textbar \ Harvard \& Smithsonian, 60 Garden St, Cambridge, MA 02138, USA}

%PENDING
\author[0000-0002-9158-7315]{R. Brahm} %rafael.brahm@uai.cl coordination, WINE board
\affiliation{Facultad de Ingenier\'{i}a y Ciencias, Universidad Adolfo Ib\'a\~nez, Av.\ Diagonal las Torres 2640, Pe\~nalol\'en, Santiago, Chile}
\affiliation{Millennium Institute for Astrophysics, Chile}

%PENDING
\author[0000-0002-5674-2404]{P. Evans}
\affiliation{El Sauce Observatory, Coquimbo Province, Chile}

%PENDING
\author[0000-0001-6588-9574]{K. A. Collins} % karenacollins@outlook.com
\affiliation{Center for Astrophysics \textbar \ Harvard \& Smithsonian, 60 Garden St, Cambridge, MA 02138, USA}

%PENDING
\author[0000-0001-8045-1765]{E. Flowers}
\affiliation{Department of Astrophysical Sciences, Princeton University, 4 Ivy Lane, Princeton, NJ 08540, USA}

%PENDING
\author[0000-0002-8964-8377]{S. N. Quinn} % squinn@cfa.harvard.edu, CHIRON vetting spectrum
\affiliation{Center for Astrophysics \textbar \ Harvard \& Smithsonian, 60 Garden St, Cambridge, MA 02138, USA}

% %PENDING
% \author{J. de Leon}
% \affiliation{Department of Astronomy, University of Tokyo, 7-3-1 Hongo, Bunkyo-ky, Tokyo 113-0033, Japan}
% 
% %PENDING
% \author[0000-0002-4881-3620]{J. Livingston}
% \affiliation{Department of Astronomy, University of Tokyo, 7-3-1 Hongo, Bunkyo-ky, Tokyo 113-0033, Japan}
%
% %PENDING
% \author[0000-0003-2989-7774]{C. Bergmann}
% \affiliation{Exoplanetary Science at UNSW, School of Physics, UNSW Sydney, NSW 2052, Australia}
% \affiliation{Deutsches Zentrum f\"ur Luft- und Raumfahrt, M\"unchener Str. 20, 82234 Wessling, Germany}

%PENDING
\author[0000-0002-3481-9052]{K. G. Stassun}
\affiliation{Vanderbilt University, Department of Physics \& Astronomy, 6301 Stevenson Center Lane, Nashville, TN 37235, USA}
\affiliation{Fisk University, Department of Physics, 1000 17th Avenue N., Nashville, TN 37208, USA}
%

%
% Princeton team
%
%PENDING
\author[0000-0002-0628-0088]{W. Bhatti}
\affiliation{Department of Astrophysical Sciences, Princeton University, 4 Ivy Lane, Princeton, NJ 08540, USA}
%
\author[0000-0002-4265-047X]{J. N. Winn}
\affiliation{Department of Astrophysical Sciences, Princeton University, 4 Ivy Lane, Princeton, NJ 08540, USA}
%
\author[0000-0001-7204-6727]{G. \'A. Bakos}
\affiliation{Department of Astrophysical Sciences, Princeton University, 4 Ivy Lane, Princeton, NJ 08540, USA}

%
% SG1 contributors
%

% % begin (ASTEP team)
% %PENDING
% \author{L. Abe} % Lyu
% \affiliation{Universit\'e C\^{o}te d'Azur, Observatoire de la C\^ote d'Azur, CNRS, Laboratoire Lagrange, Bd de l'Observatoire, CS 34229, 06304 Nice cedex 4, France}
% 
% %%PENDING->OMITTED
% %\author{A. Agabi} % Abdelkrim 
% %\affiliation{Universit\'e C\^{o}te d'Azur, Observatoire de la C\^ote d'Azur, CNRS, Laboratoire Lagrange, Bd de l'Observatoire, CS 34229, 06304 Nice cedex 4, France}
% 
% %PENDING
% \author[0000-0001-7866-8738]{N. Crouzet} % Nicolas
% \affiliation{European Space Agency, European Space Research and Technology Centre (ESA/ESTEC), Keplerlaan 1, 2201 AZ Noordwijk, The Netherlands}
% 
% %PENDING
% \author[0000-0002-3937-630X]{G. Dransfield} % Georgina
% \affiliation{School of Physics \& Astronomy, University of Birmingham, Edgbaston, Birmingham B15 2TT, United Kingdom}
% 
% \author[0000-0002-7188-8428]{T. Guillot} % Tristan 
% \affiliation{Universit\'e C\^{o}te d'Azur, Observatoire de la C\^ote d'Azur, CNRS, Laboratoire Lagrange, Bd de l'Observatoire, CS 34229, 06304 Nice cedex 4, France}
% 
% %PENDING
% \author{W. Marie-Sainte} % Wenceslas
% \affiliation{Institut Paul \'{E}mile Victor, Concordia Station, Antarctica}
% 
% %PENDING
% \author{D. M\'ekarnia} % Djamel
% \affiliation{Universit\'e C\^{o}te d'Azur, Observatoire de la C\^ote d'Azur, CNRS, Laboratoire Lagrange, Bd de l'Observatoire, CS 34229, 06304 Nice cedex 4, France}
% 
% %PENDING
% \author[0000-0002-5510-8751]{A. H.M.J. Triaud} % Amaury
% \affiliation{School of Physics \& Astronomy, University of Birmingham, Edgbaston, Birmingham B15 2TT, United Kingdom}
% % end (ASTEP team)

%
% SG2/SG3 contributors
%
\author{J.~Teske} % PFS
\affiliation{Earth and Planets Laboratory, Carnegie Institution for Science, 5241 Broad Branch Road, NW, Washington, DC 20015, USA}

\author{S.~X.~Wang}% PFS
\affiliation{Department of Astronomy, Tsinghua University, Beijing 100084, People's Republic of China}

\author{R. P. Butler}% PFS
\affiliation{Earth and Planets Laboratory, Carnegie Institution for Science, 5241 Broad Branch Road, NW, Washington, DC 20015, USA}

\author{J. D. Crane}% PFS
\affiliation{Observatories of the Carnegie Institution for Science, 813 Santa Barbara Street, Pasadena, CA 91101, USA}

\author{S. A. Shectman}% PFS
\affiliation{Observatories of the Carnegie Institution for Science, 813 Santa Barbara Street, Pasadena, CA 91101, USA}



% %PENDING
% \author[0000-0002-7595-0970]{C.~G.~Tinney}
% \affiliation{Exoplanetary Science at UNSW, School of Physics, UNSW Sydney, NSW 2052, Australia}


% %
% % WINE MPIA team
% %
% 
% %PENDING
% \author{T. Henning} %Thomas henning@mpia.de FEROS, WINE board
% \affiliation{Max-Planck-Institut f\"{u}r Astronomie, K\"onigstuhl 17, Heidelberg 69117, Germany }
% 
% %PENDING
% \author[0000-0001-9513-1449]{N. Espinoza} %N\'estor nespinoza@stsci.edu WINE board
% \affiliation{Space Telescope Science Institute, 3700 San Martin Drive, Baltimore, MD 21218, USA}
% 
% %PENDING
% \author[0000-0002-5389-3944]{A. Jord\'an} %andres.jordan@gmail.com WINE board
% \affiliation{Facultad de Ingenier\'{i}a y Ciencias, Universidad Adolfo Ib\'a\~nez, Av.\ Diagonal las Torres 2640, Pe\~nalol\'en, Santiago, Chile}
% \affiliation{Millennium Institute for Astrophysics, Chile}
% 
% %PENDING
% \author{M. Barbieri} %Maruo mauro.barbieri@uda.cl FEROS observations
% \affiliation{INCT, Universidad de Atacama, calle Copayapu 485, Copiap\'o, Atacama, Chile}
% 
% %PENDING
% \author{S. Nandakumar} %sangeetha.nandakumar@postgrados.uda.cl  FEROS observations
% \affiliation{INCT, Universidad de Atacama, calle Copayapu 485, Copiap\'o, Atacama, Chile}
% 
% %PENDING
% \author{T. Trifonov} %Trifon trifonov@mpia.de FEROS observations
% \affiliation{Max-Planck-Institut f\"{u}r Astronomie, K\"onigstuhl 17, Heidelberg 69117, Germany }
% 
% %PENDING
% \author[0000-0002-1896-2377]{J.~I.~Vines} %Jose jose.vines.l@gmail.com FEROS observations
% \affiliation{Departamento de Astronom\'ia, Universidad de Chile, Camino El Observatorio 1515, Las Condes, Santiago, Chile}
% 
% %PENDING
% \author{M. Vuckovic} %Maja maja.vuckovic@uv.cl FEROS observations
% \affiliation{Instituto de F\'isica y Astronom\'ia, Universidad de Vapara\'iso, Casilla 5030, Valpara\'iso, Chile}

%
% SG4 contributors
%
%PENDING
\author[0000-0002-0619-7639]{C.~Ziegler} % carl.ziegler@dunlap.utoronto.ca
\affiliation{Dunlap Institute for Astronomy and Astrophysics, University of Toronto, 50 St. George Street, Toronto, Ontario M5S 3H4, Canada}

%%OMITTED
%\author{C.~Brice\~{n}o} % cbriceno@ctio.noao.edu
%\affiliation{Cerro Tololo Inter-American Observatory, Casilla 603, La Serena, Chile}

%PENDING
\author{N.~Law} % nlaw@unc.edu
\affiliation{Department of Physics and Astronomy, The University of North Carolina at Chapel Hill, Chapel Hill, NC 27599-3255, USA}

%PENDING
\author[0000-0003-3654-1602]{A.~W.~Mann} % awmann@unc.edu
\affiliation{Department of Physics and Astronomy, The University of North Carolina at Chapel Hill, Chapel Hill, NC 27599-3255, USA}

%PENDING
\author[0000-0002-2532-2853]{S. B. Howell} % non-detection, but write & inquire
\affiliation{NASA Ames Research Center, Moffett Field, CA 94035, USA}

%PENDING
\author[0000-0001-7233-7508]{R. A. Matson}
\affiliation{U.S. Naval Observatory, Washington, DC 20392, USA}
%




% 
%-------------------------------------
% TESS Mission Architects:
% These authors should be listed in this order
% see https://spacebook.mit.edu/pages/viewpage.action?pageId=24543276
%-------------------------------------
%
%PENDING
\author{G. R. Ricker} % grr@space.mit.edu
\affiliation{Department of Physics and Kavli Institute for Astrophysics and Space Research, Massachusetts Institute of Technology, Cambridge, MA 02139, USA}
%
%PENDING
\author[0000-0001-6763-6562]{R. Vanderspek} % roland@space.mit.edu
\affiliation{Department of Physics and Kavli Institute for Astrophysics and Space Research, Massachusetts Institute of Technology, Cambridge, MA 02139, USA}
%
%PENDING
\author[0000-0001-9911-7388]{D. W.~Latham} % dlatham@cfa.harvard.edu
\affiliation{Center for Astrophysics \textbar \ Harvard \& Smithsonian, 60 Garden St, Cambridge, MA 02138, USA}
%
%PENDING
\author{S. Seager} % seager@mit.edu
\affiliation{Department of Earth, Atmospheric, and Planetary Sciences, Massachusetts Institute of Technology, Cambridge, MA 02139, USA}
%
%PENDING
\author[0000-0002-4715-9460]{J. M.~Jenkins} % jon.jenkins@nasa.gov
\affiliation{NASA Ames Research Center, Moffett Field, CA 94035, USA}

%
%-------------------------------------
% 3 representatives of each of SPOC, POC, TSO, for a total of 9. 
%These 9 authors should be listed in alphabetical order
%-------------------------------------


% %PENDING
% \author[0000-0002-7754-9486]{C.~J.~Burke} % cjburke@mit.edu
% \affiliation{Department of Physics and Kavli Institute for Astrophysics and Space Research, Massachusetts Institute of Technology, Cambridge, MA 02139, USA}
% 
% %PENDING
% \author[0000-0003-2313-467X]{D.~Dragomir}
% \affiliation{Department of Physics and Astronomy, University of New Mexico, Albuquerque, NM, USA}
% 	
% %%PENDING->OMITTED
% %\author[0000-0003-0918-7484]{C.~X.~Huang}
% %\affiliation{Department of Physics and Kavli Institute for Astrophysics and Space Research, Massachusetts Institute of Technology, Cambridge, MA 02139, USA}
% 
% %%PENDING->OMITTED
% %\author{R.~C.~Kidwell, Jr.} % rkidwell@stsci.edu
% %\affiliation{Space Telescope Science Institute, 3700 San Martin Drive, Baltimore MD 21218 }
% 
% %PENDING
% \author[0000-0001-8172-0453]{A.~M.~Levine} % aml@space.mit.edu
% \affiliation{Department of Physics and Kavli Institute for Astrophysics and Space Research, Massachusetts Institute of Technology, Cambridge, MA 02139, USA}
% 
% %PENDING
% \author{E.~V.~Quintana} % elisa.quintana@nasa.gov
% \affiliation{NASA Goddard Space Flight Center, 8800 Greenbelt Road, Greenbelt, MD 20771, USA}
% 
% %PENDING
% \author[0000-0001-8812-0565]{J.~E.~Rodriguez}
% \affiliation{Center for Astrophysics \textbar \ Harvard \& Smithsonian, 60 Garden St, Cambridge, MA 02138, USA}
% 
% %PENDING
% \author[0000-0002-6148-7903]{J. C. Smith} % jeffrey.c.smith-1@nasa.gov
% \affiliation{NASA Ames Research Center, Moffett Field, CA 94035, USA}
% \affiliation{SETI Institute, Mountain View, CA 94043, USA}
% 
% %PENDING
% \author[0000-0002-5402-9613]{B. Wohler} % bill.wohler@nasa.gov
% \affiliation{NASA Ames Research Center, Moffett Field, CA 94035, USA}
% \affiliation{SETI Institute, Mountain View, CA 94043, USA}


\begin{abstract}
  We report the discovery and confirmation of a hot Jupiter,
  TOI\,1937Ab, and present the evidence for and against its youth
  ($\approx$120\,Myr).
  We found the planet in images taken by the NASA TESS mission
  using the CDIPS pipeline.
  We measured its mass (1.X Mjup) and orbital obliquity
  (0$\pm$XX$^\circ$) using PFS at Magellan.
  Gaia kinematics suggest that the star is in the halo of NGC\,2516.
  The possible youth is corroborated by the {\bf G2V} star's 6.X day
  rotation period, the planet's 22.X hour orbital period, and the
  star's metallicity ([Fe/H] X.X$\pm$Y.Y dex) being consistent with
  that of the cluster (X.X$\pm$Z.Z dex).
  However, the star's spectrum does not show lithium, which argues
  against it being truly young.
  We outline tests that could support or refute the youth of the
  system, and outline the reasons why few, if any, sub-100 Myr hot
  Jupiters have been securely detected.
\end{abstract}

\keywords{
	Exoplanets (498),
  Transits (1711),
	Exoplanet evolution (491),
	Stellar ages (1581),
	Young star clusters (1833)
}

%%%%%%%%%%%%%%%%%%%%%%%%%%%%%%%%%%%%%%%%%%%%%%%%%%%%%%%%%%%%%%%%%%%%%%%%%%%%%%%


\section{Introduction}

Depending on the process that produces them, hot Jupiters arrive on
their tiny orbits on timescales of anywhere between megayears and
gigayears (CITE, CITE, CITE).
One way to distinguish between these processes is to find
young ($\lesssim$100\,Myr) hot Jupiters.
However the search for such hot Jupiters has yielded remarkably few,
if any, secure detections.
What is the youngest hot Jupiter known?

Transits have provided a pile of tantalizing candidates.  HIP\,67522b,
with an age, orbital period, and size of XX\,Myr, X.X\,days, and
TOOSMALL\,$R_\oplus$ respectively, comes close
\citep{rizzuto_tess_2020}.  Its
size however is smaller than expected for a Jovian-mass object given
its age, by a factor of $\approx1.5\times$ (CITE Burrows, Fortney,
Thorngren models, CITE Owen 2020 entropy).  A mass measurement is
therefore needed to resolve its status as either a hot Jupiter, or an
inflated Neptunian world.  No other planet comes as close to fitting
the bill.  V1298 Tau b, part of a unique system with at least four
transiting planets, is appropriately large and young. However its orbital
period (XX\,days) is too long to be a hot Jupiter (CITE David 2019,
2020).  The dips in PTFO 8-8695, long-interpreted as a candidate hot
Jupiter, are chromatic and show a changing orbital phase.  This is
inconsistent with a planetary interpretation (van Eyken 2012, Yu+15,
Onitsuka+17, Tanimoto+20, Bouma+20).

Reports of very young hot Jupiters have come from the radial velocity
(RV) technique, but these reports are difficult to verify.  TW Hya
(age=X\,Myr), for instance, showed a radial velocity semi-amplitude of
$\approx$200\,m\,s$^{-1}$ with a period of 3.6\,days, as measured in
optical spectra \citep{setiawan_young_2008} (CITEP ALSO age reference).
Even though \citet{setiawan_young_2008} observed no significant
correlation between activity indicators and the RV signal, subsequent
infrared velocities measured by \citet{huelamo_tw_2008} showed small
variations $\lesssim 35\,$m\,s$^{-1}$.  \citet{huelamo_tw_2008}
ultimately concluded that activity-induced variations were the source
of the optical signal.  More recently, very young hot Jupiters have
been reported around CI Tau, V830 Tau, and TAP 26
\citep{johns-krull_CI_Tau_candidate_2016,donati_hot_2016,donati_hot_2017,yu_hot_2017,biddle_k2_2018,flagg_co_2019}.
The planetary nature of at least two of these signals (CI Tau and V830
Tau) has been debated \citep{donati_magnetic_2020,damasso_gaps_2020}.
More generally, the Bayesian model comparison problem of ``stellar
activity alone'' versus ``stellar activity plus planet'' has been
shown to require significant amounts of both data and statistical
machinery \citep[{\it
e.g.},][]{barragan_radial_2019,klein_simulated_2020} To date, none of
the reported young hot Jupiter detections have performed this style of
analysis.  Such an analysis, or else the acquisition of multi-color
radial velocities, seems like a requirement given the challenges of
detecting small signals in the presence of significant stellar
variability.

%TODO: make figure
So where does that leave the search for young hot Jupiters?
A quick query to the NASA Exoplanet Archive yields the current state
of the search, showcased in Figure~\ref{fig:rp_vs_age}.
(FIGURE: Rp vs Age, and Mp\,sini vs Age).
The youngest transiting hot Jupiter ($R_p>R_{\rm Jup}$, $P<10\,{\rm
days}$) appears to be Qatar-4b. 
Discovered by (CITE), the age was reported to be very low because of
gyrochrones.
Not great (CITE CITE).




Section~\ref{sec:observations} describes the identification of the
candidate, and the follow-up observations that led to its confirmation.

The planet can only be understood with respect to its (putative)
host cluster, so in turn we analyze the
available six-dimensional positions and kinematics (Section~\ref{sec:gaia6d}),
the rotation periods of stars in \cn\ 
(Section~\ref{sec:rotation}), and the available lithium measurements
(Section~\ref{sec:lithium}).
We synthesize this data in
Section~\ref{sec:system} to present our best interpretation of the
system itself, in turn presenting our interpretation of the
cluster
(Section~\ref{subsec:cluster}), the star (Section~\ref{subsec:star})
and the planet (Section~\ref{subsec:planet}).  We conclude in
Section~\ref{sec:discussion} by discussing which questions we have
been able to answer, and which remain open.



\section{Identification and Follow-up Observations}
\label{sec:observations}


  \subsection{TESS Photometry}
  \label{subsec:tess}
  TESS: S7 + S9

  \subsection{Gaia Astrometry and Imaging}
  \label{subsec:gaia}
  Gaia:
    Within 20 arcsec, there are three sources. 

  % 5489726768531119616 (target, G=13.02, plx=2.38 +/- 0.017 mas, so Bp-Rp=1.00).
  %   Note RV = 27.77km/s, +/- 9.99 km/s. Quotes E(Bp-Rp) = 0.1925

  % 5489726768531118848 (G=17.59, Bp=17.7, Rp=16.2, Bp-Rp=1.502,...
  %   but has plx=2.32 +/- 0.118 mas, and VERY SIMILAR proper motions. So, it's a
  %   binary. Presumably this is the companion that Ziegler found.)

  % There's also 5489726768531122560, a G=19.5 companion at further sep.

  % ...

  % Within 30 arcsec, there are... 8 sources. The others don't have parallaxes or
  % proper motions of significant interest.

  \subsection{High-Resolution Imaging}
  \label{subsec:speckle}
  Howell et al took some, for me, and had a non-detection.

  Also, Ziegler et al took some.
    They get the 2.1 arcsec separation, faint companion.

  \subsection{Ground-based Time-Series Photometric Follow-up}
  \label{subsec:groundphot}
  LCOGT
    5 light curves. 3 i band, 1 g band, 1 z band
  El Sauce
    1 r band.

  \subsection{Spectroscopic Follow-up}
  \label{subsec:spectra}

  \subsubsection{SMARTS 1.5$\,$m / CHIRON}
  \label{subsec:chiron}
  CHIRON x1 recon 2020-02-04, another circa 2021.

  \subsubsection{PFS}
  PFS template + RVs

  %%% \begin{deluxetable}{} command tell LaTeX how many columns
%% there are and how to align them.
\startlongtable
\begin{deluxetable}{llll}
    
%% Keep a portrait orientation

%% Over-ride the default font size
%% Use Default (12pt)
\tabletypesize{\scriptsize}

%% Use \tablewidth{?pt} to over-ride the default table width.
%% If you are unhappy with the default look at the end of the
%% *.log file to see what the default was set at before adjusting
%% this value.

%% This is the title of the table.
\tablecaption{\tn\ radial velocities.}
\label{tab:rvs}

%% This command over-rides LaTeX's natural table count
%% and replaces it with this number.  LaTeX will increment 
%% all other tables after this table based on this number
%\tablenum{4}

%% The \tablehead gives provides the column headers.  It
%% is currently set up so that the column labels are on the
%% top line and the units surrounded by ()s are in the 
%% bottom line.  You may add more header information by writing
%% another line between these lines. For each column that requries
%% extra information be sure to include a \colhead{text} command
%% and remember to end any extra lines with \\ and include the 
%% correct number of &s.
\tablehead{
  \colhead{Time [BJD$_\mathrm{TDB}$]} &
  \colhead{RV [m$\,$s$^{-1}$]} &
  \colhead{$\sigma_{\rm RV}$ [m$\,$s$^{-1}$]} & 
  \colhead{Instrument}
}

%% All data must appear between the \startdata and \enddata commands
% Source:
% /Users/luke/Dropbox/proj/timmy/results/paper_tables/TOI837_rv_data.tex
\startdata
 8669.533150 &  -57.8 &    27.5 &   FEROS \\
 8669.540450 &  -13.9 &    29.4 &   FEROS \\
 8676.506930 &    6.7 &    37.8 &   FEROS \\
 8677.519150 &  -70.3 &    44.6 &   FEROS \\
 8884.787630 &  240.0 &    28.0 &  CHIRON \\
 8891.891180 &  -76.0 &    37.0 &  CHIRON \\
 8898.735330 &  -10.0 &    43.0 &  CHIRON \\
 8903.725760 &  -25.0 &    38.0 &  CHIRON \\
 8904.739930 &   80.1 &    24.5 &   FEROS \\
 8905.793630 &   88.0 &    21.7 &   FEROS \\
 8908.762520 &   45.3 &    28.3 &   FEROS \\
 8909.702140 &    0.0 &    31.8 &   FEROS \\
 8912.606750 &   41.3 &    24.1 &   FEROS \\
 8913.740580 &  161.1 &    37.3 &   FEROS \\
 8915.762170 &   10.0 &    33.0 &  CHIRON \\
 8916.714540 &  -93.5 &    33.6 &   FEROS \\
 8917.765720 & -159.7 &    24.8 &   FEROS \\
 8920.706100 &   99.0 &    32.0 &  CHIRON \\
 8922.845800 & -148.3 &    54.9 &   FEROS \\
 8915.924027 &   37.5 &   725.9 &  Veloce \\
 8921.284950 &  105.9 &   453.2 &  Veloce \\
 8922.733572 & -195.9 &   195.6 &  Veloce \\
 8924.583708 &   -7.6 &   262.3 &  Veloce \\
 8926.365810 &   14.3 &   442.6 &  Veloce \\
 8927.318146 &  207.0 &   505.2 &  Veloce \\
 8928.559780 &   -7.3 &   180.2 &  Veloce \\
 8930.324059 &   -2.6 &   152.0 &  Veloce \\
 8931.293091 &  -45.7 &   152.9 &  Veloce \\
 8932.065206 & -105.6 &   319.8 &  Veloce \\
\enddata

%% Include any \tablenotetext{key}{text}, \tablerefs{ref list},
%% or \tablecomments{text} between the \enddata and 
%% \end{deluxetable} commands

%% General table comment marker
\tablecomments{
Times are in units of ${\rm BJD}_{\rm TDB} - 2{,}450{,}000$.
}
\vspace{-0.9cm}
\end{deluxetable}


\section{Kinematics}

  \subsection{NGC 2516 broadly}
  \subsection{TOI 1937 specifically}
  If you back-integrate 20 Myr, it gets closer. But then, it gets
  further.

\section{Rotation}

  \subsection{NGC 2516 broadly}
  \subsection{TOI 1937 specifically}

\section{Lithium}

  \subsection{NGC 2516 broadly}
  \subsection{TOI 1937 specifically}

  \subsection{Comparison Stars That We Got Spectra For}
  They are at
  /Users/luke/Dropbox/proj/earhart/data/spectra/comparison\_stars.

  There are 6 CHIRON ones, of varying assailability.
  And 1 PFS unassailable member.

\section{System Modeling}
\label{sec:system}

\subsection{The Cluster}
\label{subsec:cluster}

\pararaph{Metallicity}
(Quoting Jilinski+09)
``After an initial period where several authors (see Terndrup et al.
2002) found for NGC 2516 a metallicity of a few tenths dex below the
solar, more recent analysis (Irwin et al. 2007; Jeffries et al.  2001;
Sciortino et al. 2001; Sung et al. 2002) places NGC 2516 with a
metallicity close to solar. ''

\subsubsection{Physical Characteristics}
\label{subsec:clusterchar}

\paragraph{Mass} 1.8Mjup

\paragraph{Obliquity} Less than 20 deg or so. Consistent with
expectations given its short orbital period (e.g., Anderson et al
2021). 

\subsubsection{HR Diagram}
\label{subsec:hr}

\subsection{The Star}
\label{subsec:star}

\subsubsection{Membership of \tn\ in \cn}
\label{subsec:member}

\subsubsection{Rotation}

\subsubsection{Lithium}

\subsubsection{Stellar Parameters}
\label{subsec:starparams}

\subsection{The Planet}
\label{subsec:planet}


\section{Discussion}
\label{sec:discussion}

\paragraph{Isn't the rotation period a little slow?}
Yes. But hot Jupiters affect the star's rotation, and it's hard to be
sure how much. Consider TOI 1431b, an Am-stype with vsini=8km/s
hosting a retrograde hot Jupiter. 
Generally speaking, retrograde systems will spin down the star (e.g.,
Anderson+2021), even if the planet is quickly ``realigned''.
The latter realignment would be needed in our case to understand TOI
1937Ab, since the observed orbit is prograde.

\subsection{The present}

The minimum orbital period beyond which a planet is tidally sheared
apart depends on the concentration of mass within the planet
(Rappaport+2013).
For a VERY CENTRALLY CONCENTRATED planet with \tn's
mean density of 0.XX g/cm3, this period is XX.X hours.
For the limit of a uniform density distribution, the period would be
XX.X hours.

is 

XX.X hr (e.g., Rappaport+ 2013).
Given ,


The Roche limit 

Roche limit for a close-orbiting planet can
be expressed as a minimum orbital period depending
on the density distribution of the planet (Rappaport
et al. 2013). For WASP-12b, with a mean density of
0.46 g cm􀀀3, the Roche-limiting orbital period is 14.2 hr
assuming the mass of the planet to be concentrated near
the center, and 18.6 hr in the opposite limit of a spher-
ical and incompressible planet. The

\subsection{The future}:
- Detailed multi-abundance modelling.
  i.e,. Get spectra of other cluster stars using PFS, and compare the metallicities.
- 2$\farc$5 separate, multiband imager. (With LCOGT, 2m in Australia,
  z-band)->Do PSF fitting.

- K-band image. MagAO? Multicolor imager.
  E.g., Keck-NIRC2 (if it were in the North).
  ???What is the prediction for the G-K color???
  Doesn't seem all that promising. G-mag would predict something like
  M2.5V or M3.5V (0.40Msun or 0.30Msun from mamajek).
  At 100 Myr, a 0.20Msun star is 18Gcm vs 16Gcm (12\% inflated;
  Burrows+01). This is not a whole lot. And it is worse for more massive
  stars.
- Maybe GROND -- has K-band. At La Silla. On 2.2m.  griz+K.
  Going whole hog, maybe SPHERE could do it.


%%%%%%%%%%%%%%%%%%%%%%%%%%%%%%%%%%%%%%%%%%%%%%%%%%%%%%%%%%%%%%%%%%%%%%%%%%%%%%%


%\clearpage
\acknowledgements
\raggedbottom

The authors thank X and Y for fruitful discussions.
%
L.G.B. and J.H. acknowledge support by the TESS GI Program, program
NUMBER, through NASA grant NUMBER.
%
This study was based in part on observations at Cerro Tololo
Inter-American Observatory at NSF's NOIRLab (NOIRLab Prop. ID
2020A-0146; 2020B-NUMBER PI: L{.}~Bouma), which is managed by the
Association of Universities for Research in Astronomy (AURA) under a
cooperative agreement with the National Science Foundation.
%
ACKNOWLEDGE PFS / CAMPANAS.
%
This paper includes data collected by the TESS mission, which are
publicly available from the Mikulski Archive for Space Telescopes
(MAST).
%
Funding for the TESS mission is provided by NASA's Science Mission
directorate.
%

% The ASTEP project acknowledges support from the French and Italian
% Polar Agencies, IPEV and PNRA, and from Universit\'e C\^ote d'Azur
% under Idex UCAJEDI (ANR-15-IDEX-01). We thank the dedicated staff at
% Concordia for their continuous presence and support throughout the
% Austral winter.
% %
% This research received funding from the European Research Council
% (ERC) under the European Union's Horizon 2020 research and innovation
% programme (grant n$^\circ$ 803193/BEBOP), and from the
% Science and Technology Facilities Council (STFC; grant n$^\circ$
% ST/S00193X/1).
%
% The Digitized Sky Survey was produced at the Space Telescope Science
% Institute under U.S. Government grant NAG W-2166.
% Figure~\ref{fig:scene} is based on photographic data obtained using
% the Oschin Schmidt Telescope on Palomar Mountain.
%

This research was based in part on observations obtained at the
Southern Astrophysical Research (SOAR) telescope, which is a joint
project of the Minist\'{e}rio da Ci\^{e}ncia, Tecnologia e
Inova\c{c}\~{o}es (MCTI/LNA) do Brasil, the US National Science
Foundation's NOIRLab, the University of North Carolina at Chapel Hill
(UNC), and Michigan State University (MSU).

This research made use of the Exoplanet Follow-up Observation
Program website, which is operated by the California Institute of
Technology, under contract with the National Aeronautics and Space
Administration under the Exoplanet Exploration Program.
% %
% This research made use of the NASA Exoplanet Archive, which is
% operated by the California Institute of Technology, under contract
% with the National Aeronautics and Space Administration under the
% Exoplanet Exploration Program.
% %

This research made use of the SVO Filter Profile Service
(\url{http://svo2.cab.inta-csic.es/theory/fps/}) supported from the Spanish
MINECO through grant AYA2017-84089.

Resources supporting this work were provided by the NASA High-End
Computing (HEC) Program through the NASA Advanced Supercomputing (NAS)
Division at Ames Research Center for the production of the SPOC data
products.
%

% A.J.\ and R.B.\ acknowledge support from project IC120009 ``Millennium
% Institute of Astrophysics (MAS)'' of the Millenium Science Initiative,
% Chilean Ministry of Economy. A.J.\ acknowledges additional support
% from FONDECYT project 1171208.  J.I.V\ acknowledges support from
% CONICYT-PFCHA/Doctorado Nacional-21191829.  R.B.\ acknowledges support
% from FONDECYT Post-doctoral Fellowship Project 3180246.
% %
% C.T.\ and C.B\ acknowledge support from Australian Research Council
% grants LE150100087, LE160100014, LE180100165, DP170103491 and
% DP190103688.
% %
% C.Z.\ is supported by a Dunlap Fellowship at the Dunlap Institute for
% Astronomy \& Astrophysics, funded through an endowment established by
% the Dunlap family and the University of Toronto.
% %
% D.D.\ acknowledges support through the TESS Guest Investigator Program
% Grant 80NSSC19K1727.
%
%
%
% %
% Based on observations obtained at the Gemini Observatory, which is
% operated by the Association of Universities for Research in Astronomy,
% Inc., under a cooperative agreement with the NSF on behalf of the
% Gemini partnership: the National Science Foundation (United States),
% National Research Council (Canada), CONICYT (Chile), Ministerio de
% Ciencia, Tecnolog\'{i}a e Innovaci\'{o}n Productiva (Argentina),
% Minist\'{e}rio da Ci\^{e}ncia, Tecnologia e Inova\c{c}\~{a}o (Brazil),
% and Korea Astronomy and Space Science Institute (Republic of Korea).
% %
% Observations in the paper made use of the High-Resolution Imaging
% instrument Zorro at Gemini-South. Zorro was funded by the NASA
% Exoplanet Exploration Program and built at the NASA Ames Research
% Center by Steve B. Howell, Nic Scott, Elliott P. Horch, and Emmett
% Quigley.
% %
% This research has made use of the VizieR catalogue access tool, CDS,
% Strasbourg, France. The original description of the VizieR service was
% published in A\&AS 143, 23.
% %
% This work has made use of data from the European Space Agency (ESA)
% mission {\it Gaia} (\url{https://www.cosmos.esa.int/gaia}), processed
% by the {\it Gaia} Data Processing and Analysis Consortium (DPAC,
% \url{https://www.cosmos.esa.int/web/gaia/dpac/consortium}). Funding
% for the DPAC has been provided by national institutions, in particular
% the institutions participating in the {\it Gaia} Multilateral
% Agreement.
%
% (Some of) The data presented herein were obtained at the W. M. Keck
% Observatory, which is operated as a scientific partnership among the
% California Institute of Technology, the University of California and
% the National Aeronautics and Space Administration. The Observatory was
% made possible by the generous financial support of the W. M. Keck
% Foundation.
% The authors wish to recognize and acknowledge the very significant
% cultural role and reverence that the summit of Maunakea has always had
% within the indigenous Hawaiian community.  We are most fortunate to
% have the opportunity to conduct observations from this mountain.
%
% \newline
%

\software{
  \texttt{arviz} \citep{arviz_2019},
  \texttt{astrobase} \citep{bhatti_astrobase_2018},
  %\texttt{astroplan} \citep{astroplan2018},
	\texttt{AstroImageJ} \citep{collins_astroimagej_2017},
  \texttt{astropy} \citep{astropy_2018},
  \texttt{astroquery} \citep{astroquery_2018},
  %\texttt{BATMAN} \citep{kreidberg_batman_2015},
  \texttt{ceres} \citep{brahm_2017_ceres},
  \texttt{cdips-pipeline} \citep{bhatti_cdips-pipeline_2019},
  \texttt{corner} \citep{corner_2016},
  %\texttt{emcee} \citep{foreman-mackey_emcee_2013},
  \texttt{exoplanet} \citep{exoplanet:exoplanet}, and its
  dependencies \citep{exoplanet:agol20, exoplanet:kipping13, exoplanet:luger18,
  	exoplanet:theano},
	%\texttt{IDL Astronomy User's Library} \citep{landsman_1995},
  \texttt{IPython} \citep{perez_2007},
	%\texttt{isochrones} \citep{morton_2015_isochrones},
	%\texttt{lightkurve} \citep{lightkurve_2018},
  \texttt{matplotlib} \citep{hunter_matplotlib_2007}, 
  %\texttt{MESA} \citep{paxton_modules_2011,paxton_modules_2013,paxton_modules_2015}
  \texttt{numpy} \citep{walt_numpy_2011}, 
  \texttt{pandas} \citep{mckinney-proc-scipy-2010},
  \texttt{pyGAM} \citep{serven_pygam_2018_1476122},
  \texttt{PyMC3} \citep{salvatier_2016_PyMC3},
  \texttt{radvel} \citep{fulton_radvel_2018},
  %\texttt{scikit-learn} \citep{scikit-learn},
  \texttt{scipy} \citep{jones_scipy_2001},
  \texttt{tesscut} \citep{brasseur_astrocut_2019},
	%\texttt{VESPA} \citep{morton_efficient_2012,vespa_2015},
  %\texttt{webplotdigitzer} \citep{rohatgi_2019},
  \texttt{wotan} \citep{hippke_wotan_2019}.
}
\ 

\facilities{
 	{\it Astrometry}:
 	Gaia \citep{gaia_collaboration_gaia_2016,gaia_collaboration_gaia_2018}.
 	{\it Imaging}:
    Second Generation Digitized Sky Survey,
    SOAR~(HRCam; \citealt{tokovinin_ten_2018}).
 	%Keck:II~(NIRC2; \url{www2.keck.hawaii.edu/inst/nirc2}).
 	%Gemini:South~(Zorro; \citealt{scott_nessi_2018}.
 	{\it Spectroscopy}:
	CTIO1.5$\,$m~(CHIRON; \citealt{tokovinin_chironfiber_2013}),
  PFS ({\bf CITE}),
  %  MPG2.2$\,$m~(FEROS; \citealt{kaufer_commissioning_1999}),
	AAT~(Veloce; \citealt{gilbert_veloce_2018}).
 	%Keck:I~(HIRES; \citealt{vogt_hires_1994}).
 	%{\bf VLT (number), UVES and GIRAFFE} (CITE: Pasquini et al 2002)
% 	Euler1.2m~(CORALIE),
% 	ESO:3.6m~(HARPS; \citealt{mayor_setting_2003}).
 	{\it Photometry}:
%	  ASTEP:0.40$\,$m (ASTEP400),
% 	CTIO:1.0m (Y4KCam),
% 	Danish 1.54m Telescope,
	  El Sauce:0.356$\,$m,
% 	Elizabeth 1.0m at SAAO,
% 	Euler1.2m (EulerCam),
% 	Magellan:Baade (MagIC),
% 	Max Planck:2.2m	(GROND; \citealt{greiner_grond7-channel_2008})
% 	NTT,
% 	SOAR (SOI),
 	TESS \citep{ricker_transiting_2015}.
% 	TRAPPIST \citep{jehin_trappist_2011},
% 	VLT:Antu (FORS2).
}

% \input{TOI837_phot_table.tex}
% %% \begin{deluxetable}{} command tell LaTeX how many columns
%% there are and how to align them.
\startlongtable
\begin{deluxetable}{llll}
    
%% Keep a portrait orientation

%% Over-ride the default font size
%% Use Default (12pt)
\tabletypesize{\scriptsize}

%% Use \tablewidth{?pt} to over-ride the default table width.
%% If you are unhappy with the default look at the end of the
%% *.log file to see what the default was set at before adjusting
%% this value.

%% This is the title of the table.
\tablecaption{\tn\ radial velocities.}
\label{tab:rvs}

%% This command over-rides LaTeX's natural table count
%% and replaces it with this number.  LaTeX will increment 
%% all other tables after this table based on this number
%\tablenum{4}

%% The \tablehead gives provides the column headers.  It
%% is currently set up so that the column labels are on the
%% top line and the units surrounded by ()s are in the 
%% bottom line.  You may add more header information by writing
%% another line between these lines. For each column that requries
%% extra information be sure to include a \colhead{text} command
%% and remember to end any extra lines with \\ and include the 
%% correct number of &s.
\tablehead{
  \colhead{Time [BJD$_\mathrm{TDB}$]} &
  \colhead{RV [m$\,$s$^{-1}$]} &
  \colhead{$\sigma_{\rm RV}$ [m$\,$s$^{-1}$]} & 
  \colhead{Instrument}
}

%% All data must appear between the \startdata and \enddata commands
% Source:
% /Users/luke/Dropbox/proj/timmy/results/paper_tables/TOI837_rv_data.tex
\startdata
 8669.533150 &  -57.8 &    27.5 &   FEROS \\
 8669.540450 &  -13.9 &    29.4 &   FEROS \\
 8676.506930 &    6.7 &    37.8 &   FEROS \\
 8677.519150 &  -70.3 &    44.6 &   FEROS \\
 8884.787630 &  240.0 &    28.0 &  CHIRON \\
 8891.891180 &  -76.0 &    37.0 &  CHIRON \\
 8898.735330 &  -10.0 &    43.0 &  CHIRON \\
 8903.725760 &  -25.0 &    38.0 &  CHIRON \\
 8904.739930 &   80.1 &    24.5 &   FEROS \\
 8905.793630 &   88.0 &    21.7 &   FEROS \\
 8908.762520 &   45.3 &    28.3 &   FEROS \\
 8909.702140 &    0.0 &    31.8 &   FEROS \\
 8912.606750 &   41.3 &    24.1 &   FEROS \\
 8913.740580 &  161.1 &    37.3 &   FEROS \\
 8915.762170 &   10.0 &    33.0 &  CHIRON \\
 8916.714540 &  -93.5 &    33.6 &   FEROS \\
 8917.765720 & -159.7 &    24.8 &   FEROS \\
 8920.706100 &   99.0 &    32.0 &  CHIRON \\
 8922.845800 & -148.3 &    54.9 &   FEROS \\
 8915.924027 &   37.5 &   725.9 &  Veloce \\
 8921.284950 &  105.9 &   453.2 &  Veloce \\
 8922.733572 & -195.9 &   195.6 &  Veloce \\
 8924.583708 &   -7.6 &   262.3 &  Veloce \\
 8926.365810 &   14.3 &   442.6 &  Veloce \\
 8927.318146 &  207.0 &   505.2 &  Veloce \\
 8928.559780 &   -7.3 &   180.2 &  Veloce \\
 8930.324059 &   -2.6 &   152.0 &  Veloce \\
 8931.293091 &  -45.7 &   152.9 &  Veloce \\
 8932.065206 & -105.6 &   319.8 &  Veloce \\
\enddata

%% Include any \tablenotetext{key}{text}, \tablerefs{ref list},
%% or \tablecomments{text} between the \enddata and 
%% \end{deluxetable} commands

%% General table comment marker
\tablecomments{
Times are in units of ${\rm BJD}_{\rm TDB} - 2{,}450{,}000$.
}
\vspace{-0.9cm}
\end{deluxetable}

% \input{ic2602_ages.tex}
\begin{table*}
\scriptsize
\setlength{\tabcolsep}{2pt}
\centering
\caption{Literature and Measured Properties for TOI$\,$1937A}
\label{tab:starparams}
%\tablenum{2}
\begin{tabular}{llcc}
  \hline
  \hline
Other identifiers\dotfill & \\
\multicolumn{3}{c}{TIC 268301217} \\
\multicolumn{3}{c}{GAIADR2 5489726768531119616} \\
\multicolumn{3}{c}{GAIAEDR3 5489726768531119616} \\
\hline
\hline
Parameter & Description & Value & Source\\
\hline 
$\alpha_{J2016.0}$\dotfill	&Right Ascension (deg)\dotfill & 116.3707 $\pm$ 0.0109& 1	\\
$\delta_{J2016.0}$\dotfill	&Declination (deg)\dotfill & -52.3833 $\pm$ 0.0097  & 1	\\
$l_{J2016.0}$\dotfill	&Galactic Longitude (deg)\dotfill & 265.3082 & 1	\\
$b_{J2016.0}$\dotfill	&Galactic Latitude (deg)\dotfill & -13.5487 & 1	\\
%\\
%$NUV$\dotfill           & GALEX $NUV$ mag.\dotfill & 13.804 $\pm$ 0.004 & 2 \\
%$FUV$\dotfill           & GALEX $FUV$ mag.\dotfill & 18.466 $\pm$ 0.056 & 2 \\
\\
%B\dotfill			&Johnson B mag.\dotfill & 11.119 $\pm$ 0.107		& 2	\\
V\dotfill			&Johnson V mag.\dotfill & 13.18 $\pm$ 0.10		& 2	\\
%$B$\tablenote{The uncertainties of the photometry have a systematic error floor applied. Even still, the global fit requires a significant scaling of the uncertainties quoted here to be consistent with our model, suggesting they are still significantly underestimated for one or more of the broad band magnitudes}\dotfill		& APASS Johnson $B$ mag.\dotfill	& 13.001 $\pm$	0.02& 2	\\
%$V$\dotfill		& APASS Johnson $V$ mag.\dotfill	& 11.808 $\pm$	0.02& 2	\\
%\\
${\rm G}$\dotfill     & Gaia $G$ mag.\dotfill     & 13.005$\pm$0.003 & 1\\
${\rm Bp}$\dotfill     & Gaia $Bp$ mag.\dotfill     & 13.417 $\pm$0.003 & 1\\
${\rm Rp}$\dotfill     & Gaia $Rp$ mag.\dotfill     & 12.421$\pm$0.004 & 1\\
${\rm T}$\dotfill     & TESS mag.\dotfill     & 12.493$\pm$0.006 & 2\\
%$u'$\dotfill        & Sloan $u'$ mag.\dotfill & 14.706 $\pm$ 0.006& 3\\
%$g'$\dotfill		& APASS Sloan $g'$ mag.\dotfill	& 12.407 $\pm$ 0.02	&  2	\\
%$r'$\dotfill		& APASS Sloan $r'$ mag.\dotfill	& 11.311 $\pm$ 0.02	&  2	\\
%$i'$\dotfill		& APASS Sloan $i'$ mag.\dotfill	& 10.927 $\pm$ 0.04 &  2	\\
%\\
J\dotfill			& 2MASS J mag.\dotfill & 11.717  $\pm$ 0.020	& 3	\\
H\dotfill			& 2MASS H mag.\dotfill & 11.324 $\pm$ 0.026	    &  3	\\
K$_{\rm S}$\dotfill			& 2MASS ${\rm K_S}$ mag.\dotfill & 11.226 $\pm$ 0.021 &  3	\\
%\\
W1\dotfill		& WISE1 mag.\dotfill & 11.135 $\pm$ 0.023 & 4	\\
W2\dotfill		& WISE2 mag.\dotfill & 11.155 $\pm$ 0.020 &  4 \\
W3\dotfill		& WISE3 mag.\dotfill & 11.160 $\pm$ 0.086& 4	\\
W4\dotfill		& WISE4 mag.\dotfill & 9.246 $\pm$ N/A &  4	\\
\\
$\pi$\dotfill & Gaia EDR3 parallax (mas) \dotfill & 2.411 $\pm$ 0.011 &  1 \\
$d$\dotfill & Distance (pc)\dotfill & $414.7 \pm 1.9$ & 1 \\
$\mu_{\alpha'}$\dotfill		& Gaia EDR3 proper motion\dotfill & -5.627 $\pm$ 0.013 & 1 \\
                    & \hspace{3pt} in RA (mas yr$^{-1}$)	&  \\
$\mu_{\delta}$\dotfill		& Gaia EDR3 proper motion\dotfill 	&  11.309 $\pm$ 0.013 &  1 \\
                    & \hspace{3pt} in DEC (mas yr$^{-1}$) &  \\
RUWE\dotfill		& Gaia EDR3 renormalized\dotfill 	&  0.908 &  1 \\
                    & \hspace{3pt} unit weight error &  \\
RV\dotfill & Gaia EDR3 systemic \hspace{9pt}\dotfill  & $17.44 \pm 0.64$$^{\dagger}$ & 1 \\
                    & \hspace{3pt} radial velocity (\kms)  & \\
RV\dotfill & Adopted systemic \hspace{9pt}\dotfill  & $17.44 \pm 0.64$$^{\dagger}$ & 1 \\
                    & \hspace{3pt} radial velocity (\kms)  & \\
%
\\
$v\sin{i_\star}$\dotfill &  Rotational velocity (\kms) \hspace{9pt}\dotfill &  -- $\pm$ -- & 5 \\
$v_{\rm mac}$\dotfill &  Macroturbulence velocity (\kms) \hspace{9pt}\dotfill &  -- $\pm$ -- & 5 \\
${\rm [Fe/H]}$\dotfill &   Metallicity \hspace{9pt}\dotfill & -- $\pm$ -- & 5 \\
$T_{\rm eff}$\dotfill &  Effective Temperature (K) \hspace{9pt}\dotfill & ---- $\pm$ --- &  6  \\
$\log{g_{\star}}$\dotfill &  Surface Gravity (cgs)\hspace{9pt}\dotfill &  x.xxx $\pm$ 0.049  &  6 \\
%
Li EW\dotfill & 6708\AA\ Equiv{.} Width (m\AA) \dotfill & $<30$  & 7 \\
%
$P_{\rm rot}$\dotfill & Rotation period (d)\dotfill & $6.5\pm X.X$  & 8 \\
Age & Adopted stellar age (Myr)\dotfill & ---  &  9 \\
% $E(B-V)$\dotfill & Reddening (mag)\dotfill & $0.06 \pm 0.02$ & 9 \\
%
Spec. Type\dotfill & Spectral Type\dotfill & 	G2V & 5 \\
%
$R_\star$\dotfill & Stellar radius ($R_\odot$)\dotfill & X.XXX$\pm$X.XXX & 6 \\
$M_\star$\dotfill & Stellar mass ($R_\odot$)\dotfill & 1.XXX$\pm$X.XXX & 6 \\
%$F_{\rm bol}$\dotfill & Stellar bolometric flux (cgs)\dotfill & (1.967$\pm$0.046)$\times10^{-9}$ & 9 \\
$A_{\rm V}$\dotfill & Interstellar reddening (mag)\dotfill & 0.XX$\pm$0.XX & 10 \\
% $U^{*}$\dotfill & Space Velocity (\kms)\dotfill & $26.24 \pm 0.46$  & \S\ref{sec:uvw} \\
% $V$\dotfill       & Space Velocity (\kms)\dotfill & $-71.52 \pm 1.68$ & \S\ref{sec:uvw} \\
% $W$\dotfill       & Space Velocity (\kms)\dotfill & $ -1.31 \pm 0.27$ & \S\ref{sec:uvw} \\
\hline
\end{tabular}
\begin{flushleft}
 \footnotesize{ \textsc{NOTE}---
$\dagger$ Systemic RV uncertainty is the standard deviation of single-transit radial velocities, as quoted in Gaia DR2. %$*$ $U$ is in the direction of the Galactic center. \\
  {\bf FIXME}
Provenances are:
$^1$\citet{gaia_collaboration_gaia_2018},
$^2$\citet{stassun_TIC8_2019},
$^3$\citet{skrutskie_tmass_2006},
$^4$\citet{wright_WISE_2010},
$^5$CHIRON spectra,
$^6$Method~2 (cluster isochrone, Section~\ref{subsec:starparams}),
$^7$FEROS spectra,
$^8$TESS light curve,
$^9$IC~2602 ages from isochrone \& lithium depletion analyses (Section~\ref{subsec:clusterchar}),
$^{10}$Method~1 (photometric SED fit, Section~\ref{subsec:starparams}).}
\end{flushleft}
\vspace{-0.5cm}
\end{table*}

\begin{table*}
\scriptsize
\setlength{\tabcolsep}{2pt}
\centering
\caption{Literature and Measured Properties for TOI$\,$1937B}
\label{tab:compparams}
%\tablenum{2}
\begin{tabular}{llcc}
  \hline
  \hline
Other identifiers\dotfill & \\
\multicolumn{3}{c}{TIC 766593811} \\
\multicolumn{3}{c}{GAIADR2 5489726768531118848} \\
\multicolumn{3}{c}{GAIAEDR3 5489726768531118848} \\
\hline
\hline
Parameter & Description & Value & Source\\
\hline 
$\alpha_{J2016.0}$\dotfill	&Right Ascension (deg)\dotfill & 116.3706 $\pm$ 0.0098& 1	\\
$\delta_{J2016.0}$\dotfill	&Declination (deg)\dotfill & -52.3826 $\pm$ 0.0753  & 1	\\
% $l_{J2015.5}$\dotfill	&Galactic Longitude (deg)\dotfill & 265.3082 & 1	\\
% $b_{J2015.5}$\dotfill	&Galactic Latitude (deg)\dotfill & -13.5487 & 1	\\
%\\
%$NUV$\dotfill           & GALEX $NUV$ mag.\dotfill & 13.804 $\pm$ 0.004 & 2 \\
%$FUV$\dotfill           & GALEX $FUV$ mag.\dotfill & 18.466 $\pm$ 0.056 & 2 \\
\\
%B\dotfill			&Johnson B mag.\dotfill & 11.119 $\pm$ 0.107		& 2	\\
%V\dotfill			&Johnson V mag.\dotfill & 13.18 $\pm$ 0.10		& 2	\\
%$B$\tablenote{The uncertainties of the photometry have a systematic error floor applied. Even still, the global fit requires a significant scaling of the uncertainties quoted here to be consistent with our model, suggesting they are still significantly underestimated for one or more of the broad band magnitudes}\dotfill		& APASS Johnson $B$ mag.\dotfill	& 13.001 $\pm$	0.02& 2	\\
%$V$\dotfill		& APASS Johnson $V$ mag.\dotfill	& 11.808 $\pm$	0.02& 2	\\
%\\
${\rm G}$\dotfill     & Gaia $G$ mag.\dotfill     & 17.653$\pm$0.003 & 1\\
${\rm Bp}$\dotfill     & Gaia $Bp$ mag.\dotfill     & 17.950 $\pm$0.098 & 1\\
${\rm Rp}$\dotfill     & Gaia $Rp$ mag.\dotfill     & 16.246 $\pm$0.015 & 1\\
${\rm T}$\dotfill     & TESS mag.\dotfill     & 16.86$\pm$0.08 & 2\\
$\Delta I_{\rm C}$\dotfill     & SOAR Cousins-I mag diff.\dotfill & 4.3$\pm$0.X & 2\\
%$u'$\dotfill        & Sloan $u'$ mag.\dotfill & 14.706 $\pm$ 0.006& 3\\
%$g'$\dotfill		& APASS Sloan $g'$ mag.\dotfill	& 12.407 $\pm$ 0.02	&  2	\\
%$r'$\dotfill		& APASS Sloan $r'$ mag.\dotfill	& 11.311 $\pm$ 0.02	&  2	\\
%$i'$\dotfill		& APASS Sloan $i'$ mag.\dotfill	& 10.927 $\pm$ 0.04 &  2	\\
%\\
% J\dotfill			& 2MASS J mag.\dotfill & 11.717  $\pm$ 0.020	& 3	\\
% H\dotfill			& 2MASS H mag.\dotfill & 11.324 $\pm$ 0.026	    &  3	\\
% K$_{\rm S}$\dotfill			& 2MASS ${\rm K_S}$ mag.\dotfill & 11.226 $\pm$ 0.021 &  3	\\
% %\\
% W1\dotfill		& WISE1 mag.\dotfill & 11.135 $\pm$ 0.023 & 4	\\
% W2\dotfill		& WISE2 mag.\dotfill & 11.155 $\pm$ 0.020 &  4 \\
% W3\dotfill		& WISE3 mag.\dotfill & 11.160 $\pm$ 0.086& 4	\\
% W4\dotfill		& WISE4 mag.\dotfill & 9.246 $\pm$ N/A &  4	\\
\\
$\pi$\dotfill & Gaia EDR3 parallax (mas) \dotfill & 2.351 $\pm$ 0.089 &  1 \\
$d$\dotfill & Distance (pc)\dotfill & $425.3 \pm 16.1$ & 1 \\
$\mu_{\alpha'}$\dotfill		& Gaia EDR3 proper motion\dotfill & -5.387 $\pm$ 0.104 & 1 \\
                    & \hspace{3pt} in RA (mas yr$^{-1}$)	&  \\
$\mu_{\delta}$\dotfill		& Gaia EDR3 proper motion\dotfill 	&  11.349 $\pm$ 0.096 &  1 \\
                    & \hspace{3pt} in DEC (mas yr$^{-1}$) &  \\
RUWE\dotfill		& Gaia EDR3 renormalized\dotfill 	&  1.120 &  1 \\
                    & \hspace{3pt} unit weight error &  \\
% RV\dotfill & Gaia EDR3 systemic \hspace{9pt}\dotfill  & $17.44 \pm 0.64$$^{\dagger}$ & 1 \\
%                     & \hspace{3pt} radial velocity (\kms)  & \\
% RV\dotfill & Adopted systemic \hspace{9pt}\dotfill  & $17.44 \pm 0.64$$^{\dagger}$ & 1 \\
%                     & \hspace{3pt} radial velocity (\kms)  & \\
%
% \\
% $v\sin{i_\star}$\dotfill &  Rotational velocity (\kms) \hspace{9pt}\dotfill &  -- $\pm$ -- & 5 \\
% $v_{\rm mac}$\dotfill &  Macroturbulence velocity (\kms) \hspace{9pt}\dotfill &  -- $\pm$ -- & 5 \\
${\rm [Fe/H]}$\dotfill &   Metallicity \hspace{9pt}\dotfill & -- $\pm$ -- & 5 \\
$T_{\rm eff}$\dotfill &  Effective Temperature (K) \hspace{9pt}\dotfill & ---- $\pm$ --- &  6  \\
$\log{g_{\star}}$\dotfill &  Surface Gravity (cgs)\hspace{9pt}\dotfill &  x.xxx $\pm$ x.xx  &  6 \\
%
Li EW\dotfill & 6708\AA\ Equiv{.} Width (m\AA) \dotfill & NaN  & 7 \\
%
$P_{\rm rot}$\dotfill & Rotation period (d)\dotfill & NaN  & 8 \\
Age & Adopted stellar age (Myr)\dotfill & ---  &  9 \\
% $E(B-V)$\dotfill & Reddening (mag)\dotfill & $0.06 \pm 0.02$ & 9 \\
%
Spec. Type\dotfill & Spectral Type\dotfill & 	M1V{\bf FIX} & 5 \\
%
$R_\star$\dotfill & Stellar radius ($R_\odot$)\dotfill & 0.XXX$\pm$X.XXX & 6 \\
$M_\star$\dotfill & Stellar mass ($R_\odot$)\dotfill & 0.XXX$\pm$X.XXX & 6 \\
%$F_{\rm bol}$\dotfill & Stellar bolometric flux (cgs)\dotfill & (1.967$\pm$0.046)$\times10^{-9}$ & 9 \\
$A_{\rm V}$\dotfill & Interstellar reddening (mag)\dotfill & 0.XX$\pm$0.XX & 10 \\
% $U^{*}$\dotfill & Space Velocity (\kms)\dotfill & $26.24 \pm 0.46$  & \S\ref{sec:uvw} \\
% $V$\dotfill       & Space Velocity (\kms)\dotfill & $-71.52 \pm 1.68$ & \S\ref{sec:uvw} \\
% $W$\dotfill       & Space Velocity (\kms)\dotfill & $ -1.31 \pm 0.27$ & \S\ref{sec:uvw} \\
\hline
\end{tabular}
\begin{flushleft}
 \footnotesize{ \textsc{NOTE}---
% $\dagger$ Systemic RV uncertainty is the standard deviation of single-transit radial velocities, as quoted in Gaia DR2. %$*$ $U$ is in the direction of the Galactic center. \\
  {\bf FIXME}
Provenances are:
$^1$\citet{gaia_collaboration_gaia_2018},
$^2$\citet{stassun_TIC8_2019},
$^3$\citet{skrutskie_tmass_2006},
$^4$\citet{wright_WISE_2010},
$^5$CHIRON spectra,
$^6$Method~2 (cluster isochrone, Section~\ref{subsec:starparams}),
$^7$FEROS spectra,
$^8$TESS light curve,
$^9$IC~2602 ages from isochrone \& lithium depletion analyses (Section~\ref{subsec:clusterchar}),
$^{10}$Method~1 (photometric SED fit, Section~\ref{subsec:starparams}).}
\end{flushleft}
\vspace{-0.5cm}
\end{table*}

% \input{model_posterior_table.tex}

\clearpage
\bibliographystyle{yahapj}                            
\bibliography{bibliography} 


\listofchanges

%\allauthors
\end{document}
